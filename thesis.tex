
\documentclass[12pt,a4paper,headsepline,bibliography=totoc,idxtotoc,DIV12,openright,twoside=true,chapterprefix=on]{scrbook} %draft zum Schluss
\renewcommand*{\chapterheadstartvskip}{\vspace*{5cm}} %rausnehmen scrbook
% optional: oneside
%\pagestyle{myheadings}

%\markright{Basics}
\addtokomafont{pageheadfoot}{\linespread{1}\selectfont}
%\usepackage[T1]{fontenc} % Windows
\usepackage{ucs}
\usepackage[rflt]{floatflt}
\usepackage[german]{babel}
\usepackage{siunitx}
\usepackage{color}
%\usepackage[latin1]{inputenc} % Windows
\usepackage[utf8x]{inputenc}
\usepackage{ulem}
\usepackage{amsmath,amssymb}
\usepackage{amsthm}
%\usepackage{mathtools}
%\usepackage[dvips]{graphicx}
\usepackage{graphicx}
%\usepackage[demo]{graphicx}
\usepackage{caption}
\usepackage{subcaption}

\usepackage{bm}

\usepackage{wrapfig}
\usepackage{sidecap}
\usepackage{multirow}
\usepackage{pgf}
\usepackage{tikz}
\usetikzlibrary{arrows,automata}
%\usepackage{pgf}
\usepackage{subfig}
\usepackage[inline]{fixme}
%\usepackage[justification=raggedright,singlelinecheck=false]{caption}
\usepackage{nicefrac}
\usepackage{setspace}
\usepackage{dcolumn}
%\usepackage{ragged2e}
\usepackage{caption}
\captionsetup{format=plain,indention=0cm}
\usepackage[sort,nocompress]{cite}
%\usepackage{achicago}
%\usepackage[authoryear,square]{natbib}
\usepackage{hyperref}
\hypersetup{colorlinks=true, citecolor=black, filecolor=black, linkcolor=black, urlcolor=black}
\onehalfspacing
\usepackage{mathpazo}
\usepackage{todonotes}
\usepackage{setspace}
\usepackage{booktabs}
\usepackage[stable]{footmisc}

\usepackage[autostyle]{csquotes}

\usepackage[
    backend=biber,
    style=authoryear-icomp,
    sortlocale=de_DE,
    natbib=true,
    url=false,
    doi=true,
    eprint=false
]{biblatex}

\addbibresource{~/bibliography/MasterThesis.bib}

\usepackage[]{hyperref}
\hypersetup{
    colorlinks=true,
}

\setlength{\emergencystretch}{1em}

\setlength{\parindent}{0pt} %Unterbindet das Einrücken

\renewcommand{\proofname}{Beweis}

\theoremstyle{definition}
\newtheorem{defi}{Definition}
\newtheorem{bem}[defi]{Bemerkung}
\newtheorem{bsp}[defi]{Beispiel}
\theoremstyle{plain}
\newtheorem{satz}[defi]{Satz}
\newtheorem{theorem}[defi]{Theorem}
\newtheorem{lem}[defi]{Lemma}
\newtheorem{kor}[defi]{Korollar}

\newcommand*\dif{\mathop{}\!\mathrm{d}}
\newcommand*\Dif{\mathop{}\!\mathrm{D}}
\newcommand{\pdn}[2][]{\frac{\partial#1}{\partial#2}}


\newcommand{\IR}{\mathbb{R}}
\newcommand{\IB}{\mathbb{B}}
\newcommand{\IN}{\mathbb{N}}
\newcommand{\IC}{\mathbb{C}}
\newcommand{\re}{\mathrm{Re}} % Realteil
\newcommand{\im}{\mathrm{Im}} % Imaginärteil
\newcommand{\Bi}{\mathrm{im}} % Bild
\newcommand{\spec}{\mathrm{spec}}
\newcommand{\Id}{\mathrm{Id}}
\newcommand{\p}{\mathrm{P}}
\newcommand{\sg}{\mathrm{sgn}}
\newcommand{\Eig}{\mathrm{Eig}}
%\newcommand{\l}{\lambda}
%\newcommand{\t}{\tau}

%\newcommand{\todo}{\fixme[inline]}
\newcommand{\rg}{\text{rank}}
\newcommand{\sgn}{\text{sgn}}
\newcommand{\M}{\text{M}}
\newcommand{\codim}{\text{codim }}
\newcommand\rP{\text{r}}
\newcommand\teMSE{\text{MSE}_{\text{test}}}
\newcommand\trMSE{\text{MSE}_{\text{train}}}
\newcommand\vMSE{\text{MSE}_{\text{val}}}
\newcommand\MCC{\text{MCC}}
\newcommand\Pvalue{\text{p-value}}
%\DeclarePairedDelimiter\set{\lbrace}{\rbrace}

\usepackage[explicit]{titlesec}
\usepackage{xcolor}
\usepackage{lipsum}% just to generate text

\titleformat{\section}[block]
  {\Large}{\thesection.~#1}{1em}{}
  \titleformat{\subsection}[block]
  {\large}{\thesubsection.~#1}{1em}{}
  %\titlespacing*{\subsection}{0pt}{15pt}{10pt}
  \titlespacing*{\section}{0pt}{25pt}{20pt}

%\colorlet{myrulecolor}{black}
\definecolor{myrulecolor}{RGB}{13,105,154}% define the color for the rules

\titleformat{\chapter}[display]
  {\normalfont\scshape\Huge}
  {\hspace*{0pt}\thechapter.~#1}
  {-15pt}
  {\hspace*{-110pt}{\color{myrulecolor}\rule{\dimexpr\textwidth+80pt\relax}{3pt}}\Huge}
\titleformat{name=\chapter,numberless}[display]
  {\normalfont\scshape\Huge}
  {\hspace*{0pt}#1}
  {-15pt}
  {\hspace*{-110pt}{\color{myrulecolor}\rule{\dimexpr\textwidth+80pt\relax}{3pt}}\Huge}
\titlespacing*{\chapter}{0pt}{0pt}{30pt}


\newenvironment{abstract}{\rightskip1in\itshape}{}
%\titleformat{\section}[hang]{\Large\scshape}{\thesection}{1em}{}

%\bibliographystyle{apalike}





\begin{document}

\makeatletter
\renewcommand*\env@cases[1][1.2]{%
  \let\@ifnextchar\new@ifnextchar
  \left\lbrace
  \def\arraystretch{#1}%
  \array{@{}l@{\quad}l@{}}%
}
\makeatother

\begin{titlepage}
       %\vspace*{1cm}
       \begin{center}
       \begin{huge}
       %\textbf{Verzweigung in Netzwerken\\[9mm]}
       \textsc{Titel Thesis auf Deutsch}
       \rule{0.9\textwidth}{0.4pt}\\
       \textsc{Titel Thesis in english\\[1.8cm}
       %\textsc{Auditory information processing in Orthopteran insects \\[2cm]}% processing of auditory mating cues .... temporal mating
       \end{huge}
       \begin{large}
	Masterarbeit\\[2cm]
% 	to obtain the academic degree\\
% 	Doctor rerum naturalium (Dr. rer. nat.)\\[2cm]

	geschrieben am Institut für Geophysik\\
	der Georg-August Universität Göttingen\\[2cm]
	%submitted to the Department of Biology, Chemistry and Pharmacy\\
% 	of Freie Universität Berlin\\[3cm]
       \end{large}
       \begin{large}
       von\\[.5cm]
       Jonas Ruebsam\\
       aus Hildesheim\\
       \vfill
       \begin{center}
       2015
       \end{center}
       \end{large}
     \end{center}
\end{titlepage}

\mbox{}
\thispagestyle{empty}
\newpage
\newpage
\pagenumbering{roman}
\thispagestyle{empty}
\vfill
\noindent \textbf{Die Arbeit wurde im Zeitraum vom XXXX 2014 bis XXXX 2015 in der Arbeitsgruppe "`Fluiddynamik"'
 unter der Betreuung von Prof. Dr. Andreas Tilgner angefertigt. }\\
%
% The research presented in this dissertation was carried out from August 2010\\
% until March 2014 at the Theoretical Neuroscience \& Neuroinformatics
% group, Freie Universität Berlin, under the supervision of Prof.~Dr.~Martin Nawrot.}\\
\vfill
\begin{tabbing}
  \hspace{3cm}\=\kill
   Erstgutachterin: \quad  Prof. Dr. Andreas Tilgner - Universität Göttingen\\
   Zweitgutachter: \quad  Prof. Dr. X Y - Universität Göttingen\\
\end{tabbing}

\newpage
\mbox{}
\thispagestyle{empty}
\newpage
% \mbox{}
% \thispagestyle{empty}
% \newpage
% thispagestyle{empty}

% \vfill
% \noindent Date of defense:
% \vfill
% \vfill
% \vfill

%%%%%%%%%%%%%%%%%%%%%%%%%%%%%%%%%%%%%%%%%%%%%%%%%%%%%%%%%%%%%%%%%%%%%%%%%%%%%%%%%%%%%%%%%%%%%%%%%%%%%%%%%%%%%%%%%%%%%%%%%%%%%%%%%%%%%%%%%%%%%%%%
%%%%%%%%%%%%%%%%%%%%%%%%%%%%%%%%%%%%%%%%%%%%%%%%%%%%%%%%%%%%%%%%%%%%%%%%%%%%%%%%%%%%%%%%%%%%%%%%%%%%%%%%%%%%%%%%%%%%%%%%%%%%%%%%%%%%%%%%%%%%%%%%
% \newpage


% \newpage
% \thispagestyle{empty}
% \vspace{3cm}
% Für meine Eltern, Helga Meckenhäuser und Wolfram Seidel...
% \newpage
% \thispagestyle{empty}
% \mbox{}
% \section*{Acknowledgements}
% First of all, I would like to thank my supervisor Martin Nawrot for the opportunity to work in his interdisciplinary and multicultural research group, for his continuous support, encouragement and openness towards new projects during the past years.
% \newline
% \newline
% I express my gratitude to my experimental collaborators: Matthias Hennig for introducing me to the fascinating world of crickets, for collecting the data for the first project and his clear thoughts for the manuscript; Stefanie Krämer for confiding me to her data from grasshoppers and her patience; Bernhard Ronacher for his ideas and support for finishing the second manuscript. Thanks for providing me with real neuroscientific data!
% \newline
% \newline
% Thanks to all my colleagues. In particular, I want to thank Farzad Farkhooi without whom I would have been stuck in the last year: thanks for the numerous encouraging discussions and helpful remarks on my second project. I also thank Jan Soelter for discussing artificial neural networks over and over again, Thomas Rost for installing packages I could not install, Chris Häusler for his relaxing mood, Michael Schmuker for taking care of Bommel, Evren Pamir for his critical view, Rinaldo Betkiewicz for not loosing control in the polish Milchbar, Joachim Haenicke for his handcreme, and Tara Dezhdar for being a great office mate.
% \newline
% \newline
% I thank Yulia Oganian and Joachim Haenicke how helped to polish this thesis during the final weeks and Florian Rau who never got tired of correcting my writings about crickets and grasshoppers.
% \newline
% \newline
% Many thanks go to my ``ladies'' from Berlin and friends from Hamburg for sharing laughter and tears. Thank you so much for your moral support and love.
% \newline
% \newline
% I dedicate this thesis to my parents Helga Meckenhäuser and Wolfram Seidel to whom I am so grateful for their endless support and encouragement.
%
%
% %irst of all, I would like to thank Martin Nawrot for supervising my PhD project.
%
% \newpage
% \mbox{}
% \thispagestyle{empty}
% \newpage
%
% \newpage
% \thispagestyle{plain}
% \noindent\textbf{This dissertation is based on the following two manuscripts:}
% \vfill
% cricket MS
% \subsubsection*{Critical song features for auditory pattern recognition in crickets}
% \textit{Authors:}\\
% Gundula Meckenhäuser$^{1}$, R.~Matthias Hennig$^{2}$, Martin P.~Nawrot$^{1}$\\
% \textit{Author contributions:}\\ %Research idea by GM, RMH, MPN. Experiments by RMH, Analysis by GM, Manuscript by GM, Revision of the manuscript by MPN, RMH
% GM, RMH, MPN conceived the reseach idea. RMH designed the experiment. \\
% GM analyzed the data. GM, MPN wrote the manuscript. RMH revised the manuscript.\\
% %\\
% %\textit{Acknowledgments:} We thank Jan Clemens, Florian Rau, Jan Sölter, and Thomas Rost for fruitful discussion.\\
% \textit{Manuscript status:}\\ Manuscript has been published in PLoS ONE: doi: 10.1371/journal.pone.0055349 %\citep{Meckenhauser2013}
%
%
%
% grasshopper MS
% \bigskip
% \subsubsection*{Decoding of calling songs and their behavioral relevance from grasshopper auditory neurons}
% \textit{Authors:} \\
% Gundula Meckenhäuser$^{1,*}$, Stefanie Krämer$^{2,*}$, Farzad Farkhooi$^{1}$, Bernhard Ronacher$^{2}$, Martin P.~Nawrot$^{1}$\\
% \textit{Author contributions:} \\
% SK, BR designed the experiments. SK carried out the experiments. GM, FF, MPN developed ideas for data analysis. SK analyzed behavioral data. GM analyzed behavioral and electrophysiological data. GM, SK, BR, MPN wrote the manuscript. FF revised the manuscript.\\
%  %E.P. designed the research, collected and analyzed the data, developed the framework
% %and the model, run the simulations, wrote the paper; J.H. developed the model and revised
%  %the paper; M.N. developed the model and revised the paper;\\
%  %\textit{Acknowledgments:} We thank Michael Schmuker for helpful comments on this work. \\
% \textit{Manuscript status:} \\
% Manuscript has been submitted to Frontiers in Systems Neuroscience on March, 14th, 2014.
%
% \vfill
% \vfill
% \noindent\textbf{Author affiliations:}\\
% $1$ Theoretical Neuroscience \& Neuroinformatics, Institute of Biology, Department of Biology, Chemistry and Pharmacy, Freie Universität Berlin\\
% $2$ Behavioural Physiology Group, Department of Biology, Humboldt-Universität zu Berlin\\
% \bigskip
% $*$ equal contribution
%
% \newpage
% \mbox{}
% \thispagestyle{empty}
%
% \newpage
% \mbox{}
% \thispagestyle{empty}
%
% \thispagestyle{plain}
% \section*{Zusammenfassung}
%
%
% Nanocomposite materials are commonly used in many different applications due to their unique combinations of material properties.
% Here, we try to understand the mechanism of fracture in nanocomposites and the influences of interfaces on fracture in order to learn
% how to separate nanoscale materials efficiently. We study multilayer systems consisting of polycrystalline titanium and amorphous zirconium
% oxide. In order to test the length scale dependent behavior of fracture, we vary the layer thicknesses from 10 nm to 100 nm. The multilayers
% are deformed by using a specially designed in-situ setup inside a TEM with a STM holder.
%
% \vspace{2cm}
% \section*{Summary}
%
% Nanocomposite materials are commonly used in many different applications due to their unique combinations of material properties.
% Here, we try to understand the mechanism of fracture in nanocomposites and the influences of interfaces on fracture in order to learn
% how to separate nanoscale materials efficiently. We study multilayer systems consisting of polycrystalline titanium and amorphous zirconium
% oxide. In order to test the length scale dependent behavior of fracture, we vary the layer thicknesses from 10 nm to 100 nm. The multilayers
% are deformed by using a specially designed in-situ setup inside a TEM with a STM holder.
%
% \vfill
% \vfill
%
%
%

%%%%%%%%%%%%%%%%%%%%%%%%%%%%%%%%%%%%%%%%%%%%%%%%%%%%%%%%%%%%%%%%%%%%%%%%%%%%%%%%%%%%%%%%%%%%%%%%%%%%%%%%%%%%%%%%%%%%%%%%%%%%%%%%%%%%%%%%%%%%%%%%
%%%%%%%%%%%%%%%%%%%%%%%%%%%%%%%%%%%%%%%%%%%%%%%%%%%%%%%%%%%%%%%%%%%%%%%%%%%%%%%%%%%%%%%%%%%%%%%%%%%%%%%%%%%%%%%%%%%%%%%%%%%%%%%%%%%%%%%%%%%%%%%%
% %%%%%%%%%%%%%%%%%%%%%%%%%%%%%%%%%%%%%%%%%%%%%%%%%%%%%%%%%%%%%%%%%%%%%%%%%%%%%%%%%%%%%%%%%%%%%%%%%%%%%%%%%%%%%%%%%%%%%%%%%%%%%%%%%%%%%%%%%%%%%%%%
% \noindent\textbf{Keywords:} \\
% insect acoustic communication, neural information processing, pattern recognition, artificial neural network, na\"{\i}ve Bayes classifier
%
 %classical conditioning, insect brain,
 %honeybee, associative learning, plasticity, neuronal computation, computational modeling
%\input{titlepage}
%\setcounter{tocdepth}{1}
\setcounter{secnumdepth}{3}
%\newpage
\setcounter{page}{1}
\addtocontents{toc}{\protect\setstretch{1.15}}
\tableofcontents

\cleardoublepage

%\newpage
%\thispagestyle{empty}
%\mbox{}
%\thispagestyle{empty}
\setcounter{page}{1}
\pagenumbering{arabic}
\chapter{Theoretical Principles}

\section{Introduction}

Prior to the development of a numerical model, it is necessary to give an exact theoretical description of the
fluid systems, which are investigated in the context of this thesis.
Hence this section contains a brief overview of the derivation and the properties of the fundamental equations of motions.
For a more detailed description, the interested reader is referred to \citep{ferziger99} on which section \ref{theorie:eqm1} is based.\\
In the second part of this chapter, the equations will be extended for new types of fluid systems.
This includes the motion of fluid in a rotating frame of reference and the Rayleigh-B\'{e}nard system.
Furthermore the physical properties of these systems will be discussed.

\section{The Equations of Motion}\label{theorie:eqm1}

At any time we examine a viscous, newtonian and incompressible fluid. The equations of motion for such a fluid can be derived by considering the conversation of
mass and momentum inside a fixed control volume $\Omega \subset \mathbb{R}^3$.
Within the fluid the momentum at the position $\vec{r} = (x, y, z)^T$  is  characterized by the velocity $\vec{u}(\vec{r}, t) = (u, v, w)^T \in \mathbb{R}^3$,
meanwhile the mass distribution is given by the density distribution$\rho(\vec{r}) \in \mathbb{R}$.

\subsection{Mass Conservation}

Let $\partial V$ be the enclosing surface and $\vec{n}$ the normal vector to the control volume $\Omega$.
For any intensive property $\phi$ the reynolds transport theorem states that
\begin{align}
    \pdn[]{t} \int_{\Omega_M} \rho \phi \dif \Omega = \pdn[]{t}\int_{\Omega} \rho \phi \dif \Omega + \int_{\partial\Omega} \rho \phi \vec{v} \vec{n} \dif S
\end{align}
where $\Omega_M$ is a control mass (CM) volume, thus the time dependent volume of a fluid element passing through $\Omega$
\footnote{For one point in time it holds that $\Omega_M = \Omega$}.

By setting $\phi = 1$ one obtains the integral form of mass conversation.

\begin{align}
    \frac{\dif}{\dif t} \int_{\Omega_M}\dif V \rho(t) =  \int_{\Omega}\dif V \frac{\dif \rho}{\dif t}  &= \underbrace{-\int_{\partial \Omega}
     \rho \vec{u}\vec{n}\dif S}_{\mathrm{Massflux} \atop \mathrm{through \ } \partial \Omega} \stackrel{\text{Gauss} \atop \text{ Law}}{=}
      -\int_\Omega \dif V \vec{\nabla}\left(\rho \vec{u}\right)
\end{align}

The differential form of the equation is obtained by applying gauss law and considering an infinitesimal small control volume.
\begin{align}
     \frac{\partial \rho}{\partial t}  + \nabla \left(\rho \vec{u}\right) &= 0
\end{align}
As we investigate an incompressible fluid, which means $\rho = \text{const.}$, we get the incompressible continuity equation
\begin{align}
     \nabla \cdot \vec{u} &= 0
\end{align}

\subsection{Momentum Conservation}

Using the same approach, but with setting $\phi = \vec{v}$, results in the integral form of the momentum equation

\begin{align}
    \label{theorie:intimpulse}
    \pdn[]{t} \int_\Omega \rho \vec{u}\dif V + \int_{\partial\Omega} \rho\vec{u}\vec{u}\cdot \vec{n} \dif S =  \sum \vec{F}_{\text{ext.}} + \sum \vec{F}_{\text{int.}}
\end{align}

In addition to the left hand side, the equation is extended by additional internal and external forces, which may act on the fluid inside the control volume.
The external forces depend on the specific system  we examine, for example the buyont or the coriolis force, whereas the internal forces
are given by the pressure and the normal and shear stresses acting on a fluid element.\\
For a newtonian fluid the internal forces can be described by the stress tensor $\bm{T}$

\begin{align}
    \sum \vec{F}_{int.} = \int_{\partial \Omega} \bm{T}\vec{n} \dif S = \int_{\Omega} \dif V \nabla \bm{T} =
     \int_{\Omega} \dif V \nabla \left(- \left(p + \frac{2}{3}\mu\nabla\vec{u} \right) + 2\mu \bm{D} \right)
\end{align}
with the static pressure $p$, the dynamic viscosity $\mu$ and the deformation tensor $\bm{D}$.
Again we apply gauss law to equation \ref{theorie:intimpulse} and consider an infinitesimal volume.
The differential form of the impulse equation, also known as Navier-Stokes equation is than given by
\footnote{The term Navier-Stokes equation is generally referred to as the complete set of equations of motion or
just the impulse equation here we use the latter convention.}
\begin{align}
    \label{theorie:eqns}
    \pdn[u]{t} + \underbrace{\left(\vec{u} \vec{\nabla}\right) \vec{u}}_{\text{I}} &= \underbrace{- \frac{1}{\rho} \nabla p + \nu \Delta \vec{u}}_{\text{II}} +\sum \vec{F}_{\text{ext.}}
\end{align}
where we introduce the kinematic viscosity by the definition $\nu = \mu/\rho$.
For an incompressible fluid the force term generated by $\bm{T}$ reduces to (II).\\
Alltogether the internal force is now given by the pressure gradient and an diffusive impulse transport proportional to $\nu$.
The non-linearity of the fluid originates through term (I), which is also denoted as the advection operator.
It basically describes the change of impulse of a fluid element when moving through the velocity field.
\footnote{For example the velocity of a fluid element will change when no forces but a velocity field is present.}
It should be noted that the set of equations is not yet solvable as the pressure variable is still undetermined.

\subsection{Initial State and Boundary Conditions}

The solution of a partial differential equation, if it exists, is always undetermined by a constant of integration.
In order to determine the temporal evolution of a fluid system, it is necessary to define its initial state and therefore
to determine one specific solution.\\
This means that for every variable we have to choose an initial condition.
With respect to a numerical solution it has to be considered that, for example an instability, is always triggered by some kind of disturbance.
Thus it might be advisable to not choose a trivial solution like a zero velocity field, but instead a solution which quickly evolves in the state
of the system one wants to achieve. Many times it is useful to add pseudorandom noise.\\
\\
Since the fluid domain is spatial restricted it is as well necessary to define the physical behavior at its boundaries.
For a fluid domain $\Omega$ with the boundary $\partial \Omega$, the following boundarys are considered, as defined in \citep{Griebel1998} and \citep{ferziger99}.

\begin{description}
    \item[No-Slip Boundaries] All velocities components are set to zero $\vec{v}|_{\partial \Omega} = 0$. The fluid is at rest on $\partial \Omega$ and no flux through
                              the boundary occurs.
                              In a more general case, this kind of condition is  also referred to as Dirichlet-Condition where $\Phi|_{\partial \Omega} = c\in\mathbb{R} $,
                              for any variable $\Phi$.

    \item[Free-Slip Boundaries] The velocity component in normal direction to the wall is set to zero, hence $\vec{n} \nabla \vec{v} = 0$ and $\vec{n}\vec{v}=0$ is required.
                                No flux through $\partial \Omega$ occurs and no friction is impossed on the fluid.

    \item[No-Flux Boundaries] For a scalar $\Phi$, the flux through the boundary is zero, hence $\vec{n}\nabla \Phi = 0$.
                              Here the general case is referred to as Neumann boundary condition, where $\vec{n}\nabla \Phi = c\in \mathbb{R}$.
                              This boundary conditions is used i.e. to avoid energy flux through the domain boundary.

    \item[Periodic boundaries] Periodic boundaries can be applied in all directions of a system. For example if the system is periodic in x-Direction with a period $L$,
                                one has to ensure that all variables match on the boundaries, that is $\Phi(x) = \Phi(x + L)$, for any variable $\Phi$.
\end{description}

\subsection{Nondimensionalization}

For many fluid systems nondimensionalization is used to further symplify the equations of motion and reduce the number of free parameters.
The approach behind this scheme is to define variable substitutions, such that the overall systems is free from any physical units.\\
After the nondimensionalization the system is described by one or more dimensionless quantity, which characterize the overall physical behaviour.
Therefore it is easier to compare numerical simulations and experimental setups to one another.
We can choose the following scales for the variables of time, position, velocity and pressure (see \cite{Kundu2012}), the nondimensional variables are denoted by an asterisk.

\begin{align}
    \text{Length:}\qquad &  \vec{x}^* = \frac{\vec{x}}{L}  & \qquad \text{Velocity:}\qquad& \vec{x}^* = \frac{\vec{v}}{V}\\
    \text{Time:}  \qquad & t^* = t \cdot \frac{V}{L}      & \qquad  \text{Pressure:}\qquad & p^* = \frac{p - p_\infty}{\rho V^2}
\end{align}

Here we choose $L$ as a length and $V$ for the velocity, as typical scales from the fluid system, we consider.
The pressure scale is set as a difference with respect to a characteristic pressure $p_\infty$.
With the above defined scales the nondimensionalized Navier-Stokes equation is given by \footnote{
From now on we ignore the * for all dimensionless variables }:

\begin{align}
    \pdn[u]{t} + \vec{u} \cdot \vec{\nabla} \vec{u} &= -\nabla p + \frac{1}{Re} \Delta \vec{u} + \vec{F}_{ext.}
\end{align}

The dynamic of the system is reduced to the dimensionless quantity $\Rey$, also referred to as the Reynolds number,
defined by (\citep{Kundu2012}):

\begin{align}
    \label{theorie:renumber}
    Re := \frac{VL}{\nu} = \frac{\rho VL}{\mu}
\end{align}

\newpage

From equation \ref{theorie:renumber}, it can be seen that the Reynolds number gives the ratio between the inertial forces $\propto VL$
to the viscous forces $\propto \nu$, of the fluid system.
This means that for a small $\Rey$ the viscous forces dominate, whereas for a large $\Rey$ we can expect advection driven, maybe even turbulent flow.\\
To illustrate the behaviour, figure \ref{theorie:re_example} exemplarly shows the airflow around a cylinder at different reynolds numbers.
TODO:\\
- bild\\
- comparison\\

\begin{figure}[!pb]
    \label{theorie:re_example}
  \centering
    \missingfigure[figwidth=\textwidth]{Flow over a cylinder for different Reynolds numbers}
\end{figure}
\newpage

\section{Rotational Fluid Dynamics}

In this section we want to extend the Navier-Stokes equations to govern the physical attributes of rotating systems,
which play an important role in the context of geophysical fluid dynamics.\\
Due to the continous acceleration acting on the fluid, these systems exhibit some fundamental different behavior, than one would expect.
One important case we want to discuss, is the propagation of inertial waves inside a stratisfied rotating fluid.\\

\subsection{Equations of Motion}

We initially consider the motion of fluid in a coordinate system (\textbf{R}), rotating relative to the intertial system (\textbf{I}) around the axis $\vec{\Omega}$.
The relation of the time derivate between the two frames of motion is given by the relation
\begin{align}
    \left.{\pdn[]{t}}\right|_{\bm{I}} = \left.\pdn[]{t}\right|_{\bm{R}} + \Omega \times
\end{align}
according to \citep{Tilgner2007}.
Applying this relation to the position vector relative to the coordinate systems \textbf{R} and \textbf{I}
and by furthermore assuming a constant rotation rate, thus $\partial_t\vec{\Omega} = 0$, yields a coordinate transformation for the acceleration.

\begin{align}
    \label{theorie:rottrafo}
    \left.\pdn[\vec{u}]{t}\right|_{\bm{I}} = \left(\left.\pdn[\vec{u}]{t}  + \underbrace{2\vec{\Omega} \times \vec{u}|_R}_{\text{I}}
    - \underbrace{\vec{\Omega} \times (\vec{\Omega} \times \vec{r|_R}}_{\text{II}})\right)\right|_{\bm{R}}
\end{align}

Hence, the transition into a rotating coordinate system introduces two additional translucent forces, the coriolis force (I)  and the centrifugal force (II).\\
By substituting expression \ref{theorie:rottrafo} into the Navier-Stokes equation, we obtain the equations of motion for the rotating system.
A further simplication can be obtained by considering that the centrifugal force is independent of the velocity field,
hence it can be written in terms of a potential $\Phi$

\begin{align}
    \Omega \times (\Omega \times \vec{r}) = - \nabla \left(\frac{1}{2}\Omega^2\vec{r}^{'2}\right) = -\Phi
\end{align}

which can be substituted into the pressure gradient by defining $p^* = p - \Phi$ \citep{tritton88}.
\newpage
Last but not least we choose the following scales to obtain a nondimensional equation

\begin{align}
    \text{Length:}\qquad &  \vec{x}^* = \frac{\vec{x}}{L}  &
    \qquad \text{Velocity:}\qquad& \vec{u}^* =  \frac{\vec{u}}{|\vec{\Omega}|L}\\
    \text{Time:}  \qquad & t^* = t \cdot |\vec{\Omega}| &
    \qquad  \text{Pressure:}\qquad & p^* = \frac{p - p_\infty}{\rho L^2{|\vec{\Omega}|}^2}
\end{align}

The final nondimensionalized form of the Navier-Stokes equation for the rotating coordinate system reads

\begin{align}
    \label{theorie:rotns}
    \pdn[u]{t}+ \left(\vec{u}  \vec{\nabla}\right) \vec{u} + 2\Omega \times \vec{u}  &= -  \nabla p + \Ekman \Delta \vec{u} + \vec{F}_{\text{ext.}}
\end{align}

with the the dimensionless quantity $\Ekman$ the Ekman-Number, which is defined by

\begin{align}
    \Ekman := \frac{\nu}{|\vec{\Omega}| L^2} \hat{=} \frac{\text{viscous forces}}{\text{coriolis forces}}
\end{align}

The Ekman number describes the ratio between viscous and coriolis forces.

\subsection{Inertial Waves}

We begin this section by briefly recapture some of the fundamental properties of mechanical waves, as quoted by \cite[p.194]{Kundu2012}.

\begin{quote}
It is perhaps not an overstatement to say that wave motion is the most basic feature
of all physical phenomena. Waves are the means by which information is transmitted
between two points in space and time, without movement of the medium across the
two points. The energy and phase of some disturbance travel during a wave motion,
but motion of the matter is generally small. Waves are generated due to the existence of
some kind of “restoring force” that tends to bring the system back to its undisturbed
state, and of some kind of “inertia” that causes the system to overshoot after the
system has returned to the undisturbed state.
\end{quote}

Hence we have to consider, that the propagation of inertial waves requires a medium which contains a state of equilibrium and
a restoring force pointing back to this state in response to a disturbance.
In the context of geophysical fluid mechanics an equilibrium state can be given by a stratification.\\
Let us consider the example of a fluid with a continous stratisfied density.
This means that in the equilibrium, the density of the fluid has to decrease
continous with the height of the system. The displacement of a fluid element
 from its resting position results in a restoring force, given by the
gravitation or the buoyant force.
\footnote{This depends on the direction of the displacement}
It can be seen therefore, that a disturbance of the equillibrium state can result in the
propagation of so-called  gravity waves \cite{Clausen2011}.
\footnote{One import case we know from everyday life are surface waves, which emerge
from the discontinous densitiy stratisfaction between two fluids, i.e. water and air \cite{Clausen2011}.}\\
We know consider a uniform rotating fluid without the presence of external forces.
At some point in time a steady state is reached, as a result of the dynamic equillibrium between
a radial stratification of the angular momentum and the pressure.
The displacement of a fluid element in radial direction now results in
an imbalance of the preserved angular momentum and the pressure.
Due to the restoring force, given by the coriolis force, an oscillation develops.
Waves of this type are denoted as inertial waves \cite{Clausen2011}.

\paragraph{Plane Inertial Waves}\mbox{}\\

We know want to examine the properties of plane inertial waves.
For this reason we neglect the non-linear advection operator and the viscous stress,
which means $\Ekman=0$, in equation \ref{theorie:rotns}. The results we present in this section
are adopted from \citep[p.185]{Greenspan1990}.\\
In case of the a linear inviscid fluid, the equations of motion are fullfiled by plane wave solutions of the form

\begin{align}
    \vec{u} = \vec{U} e^{i(\vec{k}\vec{r}  - \omega t )}, \qquad
    p = P e^{i(\vec{k}\vec{r}  - \omega t )}
\end{align}

The wave is transverse, since an insertion into the continuity equation yields $\vec{U}\vec{k} = 0$.
From the momentum equation we obtain the dispersion relation

\begin{align}
    \omega = \pm \frac{ 2\vec{\Omega}\vec{K}}{|\vec{K}|} = \pm 2|\vec{\Omega}|\cos(\Theta)
\end{align}

where $\Theta$ is the polar angle with respect to the rotation axis,
such that $\vec{K}\vec{\Omega} = |\vec{K}||\vec{\Omega}|\cos{\Theta}$
This means that an inertial wave can only exist for a wave frequency smaller than twice of the rotation rate $\vec{\Omega}$.
Even more important it is to say that an inertial wave propagates with a fixed angle $\Theta$.
The phase velocity $\vec{c}_p$ and the group velocity $\vec{c}_g$ are given by

\begin{align}
    \vec{c}_p = 2 \frac{\vec{\Omega} \vec{K} }{|\vec{K}|^3} \vec{K}, \qquad
    \vec{c}_g = \frac{2 \vec{K} \times (\vec{\Omega} \times \vec{K})}{|\vec{K}|^3}
\end{align}

It can be noted that the group velocity and therefore the energy transport is perpendicular to the phase velocity of the wave.

\paragraph{Reflection of Inertial Waves}\mbox{}\\

The reflection of an inertial wave contradict's snells law, since the propagation angle $\Theta$ is preverved upon a reflection.
Furthermore it can be shown that the following relations hold (see \cite{Beardsley1970}),

\begin{align}
\vec{\Omega}\cdot \vec{K} =  \vec{\Omega}\cdot\vec{K}^\dagger, \qquad \hat{n} \times \vec{K} = \hat{n} \times \vec{K}^\dagger
\end{align}

where $\dagger$ denotes wave number upon reflection and $\hat{n}$ corresponds to the the normal vector of the wall.
As a result inertial waves exhibit a fundamental different behavior, which we want to discuss here.

-bild mit verschiedenen fällen
-inertial waves modes attractors
-closed basin

\subsection{Properties of the Ekman number}

\newpage

\section{Rayleigh-Benard System}
Not important yet.... \\
-temperatur\\
-entdim\\
-instabilität\\
-bilder\\







\chapter{Numerische Methoden}
\section{Einleitung}
\section{GPGPU Computing mit CUDA}
\section{Finite Differenzen Verfahren}
\section{Validierung}
\section{Immersed Boundary Methoden}

\newpage

\subsection{Einleitung}
Die bisher eingeführten Methoden eignen sich um auf einfachen Geometrien Simulationen durchzuführen.
Wenn die Ränder des Simulations-Gebietes nicht mehr mit dem numerischen Gitter übereinstimmen lassen sich die in Abschnitt () eingeführten Randbedingungen nicht mehr verwenden.
Das Problem lässt sich durch die Einführung eines an den Rand angepassten Gitter umgehen (s.Abb.), führt allerdings zu einem deutlich höheren Aufwand in den numerischen Berechnungen.
Zudem ist nicht eindeutig geklärt  unter welchen weiteren Performance-Einbussen sich das neue Gitter in der GPGPU-Implementierung verwenden lässt, da Bedingungen wie
z.B. Coalesced Readings  (s.Abbschnitt x) durch die unstrukturierten Daten deulich erschwert wäre.
Eine Alternative die in dieser Arbeit verwendet wird sind  sog. Immersed Boundary Methoden.
- Beschreibung allgemein verschiedene immersed boundary methoden direkt /exp/ imp etc
- hier einfache methoden wegen gpu



\subsection{No-Slip-Boundaries}
- zunächst gehen wir nur auf den fall einer der geschwindigkeiten ein bllabla
Folgende Verfahren formel mit kraftterm
weiterer abschnitt temperatur / no-flux

\subsubsection{Volume Penalization}
-allgemeinsch beschreibung paper zitat, zitat dipl. lüllff.
-idee dämpungsterm in der dgl
-problem instabilität

\subsubsection{Direct Forcing}
-Paper quote
-vgl volume penalization warum keine instablilität
-formeln
-implementierung

\subsubsection{Direct Forcing mit Volume Fraction}
-paper quote formeln
-implementierung beispiel

\subsubsection{Direct Forcing mit Interpolation}
-paper quote formeln
-implementierung beispiel

\subsection{Methoden-Vergleich und numerische Validierung}
Blabla
\subsubsection{Validierung mit MASA}
-validierung mit masa für alle verfahren oben.. cube /evtl zylinder?
-vegl. und argumentation ränder ehh auf null.

\subsubsection{Plane Poiseuille Flow}
\subsubsection{Poiseuille Flow im Zylinder}

\subsubsection{Zusammenfassung}


\subsection{No-Flux-Boundaries}

\subsubsection{'Variable Konduktivität'}












\newpage
\thispagestyle{empty}
\mbox{}

% \newpage
% \thispagestyle{empty}
% \mbox{}l
% %\cleardoublepage


\printbibliography

% \chapter*{Danksagung}

\begin{otherlanguage}{ngerman}
  \thispagestyle{empty}





  \null\vfill
  \noindent
\end{otherlanguage}

\end{document}
