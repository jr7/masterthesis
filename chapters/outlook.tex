%\chapter*{Conclusions and Outlook}
\addchap{Conclusions and Outlook}
%\addcontentsline{toc}{chapter}{Conclusions and Outlook}

In the first part of this thesis different Immersed Boundary methods were
implemented and successfully tested with a GPU based algorithm.
The validation provided good results for No-Slip boundaries, however the implementation of velocities differing from zero at the boundaries tended to be problematic.
One concern of this validation is that only steady-state flows were used as test cases.
This approach is justified since the local error created through approximations of the boundaries
has a larger influence on the global error.
However, for further validations unsteady flows could be of interest.
The validation of the interpolation method resulted in the smallest numerical error by far.
Unfortunately this method became numerically unstable for the simulations of the librating cylinder and cone.
For possible future applications it would be important to find and eliminate the origin of this instability.

The Immersed Boundary methods introduced in this thesis were used to satisfy No-Slip boundaries, however
in many scenarios it is necessary to use No-Flux and Free-Slip boundaries.
One example is the Rayleigh-B\'{e}nard system.
In this system a fluid in a cubic or cylindric container is heated from below and cooled from above.
The resulting heat transport is purely diffusive for small temperature gradients and
becomes convective for large temperature gradients above a critical instability.
For experiments and numerical simulations of this system adiabatic side walls are used \citep{Lulff2011}.

A simple extension of the Volume-Penalization method was performed \citep{Lulff2011} and
to enable No-Flux walls for the temperature, a inhomogeneous thermal diffusivity $\kappa (x, y, z)$ was introduced.
By setting $\kappa(x, y, z) \ll 1$ in the boundary domain, a decent approximation of No-Flux walls  was achieved \citep{Lulff2011}.
A further improved version of the Volume-Penalization method for No-Flux walls can be found in \citep{Brown-Dymkoski2014}.
Here the No-Flux boundaries were obtained by introducing a forcing term of the form

\begin{align}
    \vec{f}  = -\frac{H}{\nu_c}\left( n_k\pdn[v]{x_k} - q(\vec{r}, t) \right)
\end{align}

where $\nu_c$ is a regulation parameter, $\vec{n}$ the normal and $\vec{q}$  the desired flux through the boundary.
With this implementation the temperature flux is forced to zero in direction of $\vec{n}$.
Both of these methods were implemented and a first simulation of a test case verified the numerically stability.
However, a proper validation needs to be carried out in the future.
With a validated method it would  particular interesting to simulate
the Rayleigh-B\'{e}nard system in a cylindric domain.
\clearpage
The use of Free-Slip boundaries could reduce the numerical oscillations in the simulations of the librating cone.
These oscillations are the result of the large Peclet number of the system and are generated
close to the boundaries where the velocity gradients are the largest.
In simulations of precession driven dynamos in a cube with small Ekman numbers ($\propto 10^{-5}$),
no oscillations occur when using Free-Slip boundaries \footnote{Private communication with O.Goepfert}.
A first implementation of Free-Slip boundaries was performed
by inserting diagonal walls into the fluid domain which overlap with the grid points.
On these walls the Free-Slip condition can simply be applied just like in the basic GPU algorithm,
using the mirroring method introduced in Sec. \ref{sec:cuda_boundaries}.
However, a first test of this implementation became numerically unstable which is yet to be understood.
It would be desirable to obtain a stable Free-Slip implementation for future applications.

There are some proposals regarding the basic implementation of the algorithm on the GPU.
Numerical oscillations in the density could be a concern for future simulations.
This problem seems to be quiet common for numerical computations involving first order derivatives in CFD problems.
A common approach to avoid this is the use of a staggered grid, where the velocities
are stored on the cell faces and the scalar fields in the cell center.
Alternatively, different methods exist for pressure-based solvers where the pressure is interpolated to the cell faces.
One popular method in particular is the Rhie-Chow interpolation \citep{Rhie1983}.
Different improvements of this method that could be used for a future implementation can be found in literature.

It could be of further interest to implement an unstructured cartesian grid into the GPU algorithm.
This is a very difficult task which would bring some drastic improvements in return.
The error of the Immersed Boundary methods could be decreased by using a higher resolved grid in the vicinity to the boundary.
Furthermore, this would improve the resolution of boundary layers  and reduce the oscillations resulting from the use of high Peclet numbers.

In the second part of this thesis the Immersed Boundary methods were applied to different setups of a librating cylinder,  a cone and a frustum.
The results of these simulations are in agreement with theoretical
predictions \citep{Greenspan1969} and experimental results \cite{Beardsley1970}.
An experimental study of these systems is in discussion as a possible future research project.
A possible experimental setup is already in development.
It contains a rotating table, which can be controlled over a serial interface to enable different angular velocities and acceleration rates.
A camera module is integrated into the rotating frame and could be used for the computation
of the velocity profiles with a PIV\footnote{Particle Image Velocimetry}
method.
With this setup it would be possible to validate the numerical results of this thesis.
Furthermore different excitation mechanism of inertial waves could be studied.
The study of the decay of turbulent flows in the apex of the cone is of particular interest.
%A similar numerical study has been performed in \citep{ANRD} where a oscillating grid
%was used to induction of turbulent flows.

