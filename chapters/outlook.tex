\chapter{Conclusions and Outlook}

In the first part of this thesis different Immersed Boundary methods were
implemented and successfully tested on a GPU based algorithm.
The validation showed that these methods give good results for No-Slip boundaries,
the use with velocities different from zero at the boundaries tends to be problematic.

One concern of the validation is that only steady-state flows were used as test cases.
This approach is justified since the local error created trough approximations of the boundaries
has a larger influence on the global errror.
However, for further validations unsteady flows could be of interest.

The validation results of the interpolation method showed by far the smallest numerical error.
Unfortunately this method became numerically unstable for the simulations of the librating cone.

It would be usefull to find the issue causing this numerical instabiltiy. blab lba


-no flux boundaries rayleigh benard

-free slip bonundaries no oszillations  example oli cube
-kleinerer grient an den seiten
-abeir schwieriger zu implementiern

-ip method testen no time in this thesis


ALGO:
-freeslip oder grid mit höhere auflösung am rand
-unstructured cartesian grid algorithm cuda
-poisson solver ..
-ibm methods


CONE:
-experiment
