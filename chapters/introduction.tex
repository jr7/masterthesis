\chapter{Introduction}

In geophysics the study of rotating flows is of particular importance.
The flow in a rotating fluid exhibits fundamental different properties as a result of the coriolis force.
In these systems physical solutions are of the form of inertial waves
that are created by a perturbation of the stratisfied angular momentum of the fluid.
A further extension of these fluids systems is given by a coupling to magnetic fields.
The resulting equations are used for the study of dynamos, i.e. the magnetic field of the earth.
In the fluid dynamic research group of the geophysical institute, these fluid flows are
studied numerically by simulations using GPU optimized algorithms.

The use of GPUs for scientific computing has become increasingly popular in the last years.
A vast number of applications  can obtain a massive speed-up with the use of GPUs.
High oriented frameworks like the NVIDIA CUDA API simplify the creation
and maintance of parallelized algorithms.
For example machine learning algorithms are easy to parallelize and to implement on a GPU.

However, the parallelization of CFD algorithms can become a complex task.
On a GPU a regular and homogenous memory access is required to enable a high memory bandwith.
For curved geometries the numerical solution of CFD problems is often performed
on unstructured  or structured but body conforming grids \citep{Mittal2005}.
As a result the access to memory locations is heterogenous.
Adjacent memory cells are not accessed successively.

A common approach to perform such computations on cartesian grids are Immersed Boundary methods.
The concept behind this methods is the embedding of the curved boundary into a cartesian grid.
By using additional forcing terms and intepolation methods the desired boundary condition can be fulfilled.
The current implementation of the GPU algorithm of the research group uses a cartesian grid.
Fluid flows can be simulated on cubic geometries but not curved ones.

The goal of this master thesis is to extend the existing GPU algorithm with the use of Immersed Boundary methods
to enable the simulation of flows in curved  geometries.
The extended algorithm will be than be used for the simulation of a librating cone.
In this system inertial waves are excited by a modulation of the rotation rate.
An theoretical and experimntal studies are available for a comparison \citep{1}, \citep{1}, \citep{1}.

\clearpage

The content of this  master thesis is conceptually structured into three different parts.

The first part is given by the  chapters \RN{1} to \RN{3}
In the first chapter an introduction to the theoretical concepts is given.
This includes a description of the Navier-Stokes equations and the properties of fluid flows in rotating systems.
The second chapter introduces the numerical methods, in particular the use of finite difference methods.
Furthermore numerical stability criterions and the method of artifical incompressibility is introduced.

The second part is given by the chapters \RN{4} to \RN{5}
This part introduces the Immersed Boundar Methods in chapter \RN{4}.

The third part given by chapter \RN{5}.
-librating cylinder
-librating cone and frustum

The first obective can  be described as fallow
- combine the use of gpu with ibm to enable simulation of curved geometries on a gpu
-  this thesis deals with the simulaiton of geophysical flows on  curved geometries using a gpu based algorithm (muss das an den anfang?)
- in order to test this on a physicl system  librating cone
- study the phyisical properties of this system inertial waves  and heliciy

-this thesis is build by three esction
-in the first part theorie etc
- in the second part introduction of ibm and the validaion of ibm with stanrad test cases
- in the third part a study of  librating cone with the use of immersed boundaryie methods will be performed












