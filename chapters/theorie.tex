\chapter{Theoretische Grundlagen}

\section{Einleitung}

Eine numerische Modellierung setzt zunächst eine exakte theoretische Beschreibung des strömungsmechanischen Problems vorraus.
Daher soll in diesem Abschnitt auf die theoretischen Grundlagen der Strömungsmechanik eines inkompressiblen Fluids eingegangen werden.
Auf eine exakte mathematische Herleitung wird dabei verzichtet, der interessierte Leser sei auf [Ferziger?] oder z.B. [rritton?] verwiesen.
Anschließend werden die Grundgleichungen für verschiedene Systeme erweitert, dazu gehört die Strömung in rotierenden Systemen sowie das Rayleigh-Benard-System.

\section{Die Bewegungsgleichungen}

Für die Charakterisierung einer Strömung an einem Punkt $\vec{r} = (x, y, z)^T$  im Gebiet $\Omega \subset \mathbb{R}^3$ betrachten wir zunächst
die Geschwindigkeit $\vec{u}(\vec{r}, t) = (u, v, w)^T \in \mathbb{R}^3$ und die Dichteverteilung $\rho(\vec{r}) \in \mathbb{R}$.

\subsection{Kontinuitätsgleichung}

Sei nun $V \subset \Omega$  ein abschlossenes Teilvolumen mit einem zu seiner Oberfläche $\partial V$ gegebenen normalen Vektor $\dif \vec{S}$.
Infolge der Massenerhaltung können wir annehmen, dass sich die Gesamtmasse innerhalb von V nur ändert wenn ein
Transport $\propto \vec{u}\dif \vec{S}$ in- oder aus dem Volumen stattfindet.
Mathematisch präziser formuliert und mit Verwendung des Satz von Gauß (*) folgt die integrale Form der Kontinuitätsgleichung.

\begin{align}
    \frac{\dif}{\dif t} \int_{V}\dif V \rho(t) =  \int_{V}\dif V \frac{\dif \rho}{\dif t}  &= \underbrace{-\int_{\partial V}
     \rho \vec{u}\dif \vec{S}}_{\mathrm{Massenfluss} \atop \mathrm{durch \ } \partial V} \overset{*}{=} -\int_V \dif V \vec{\nabla}\vec{u}
\end{align}

Durch Bildung des Limes $V\rightarrow 0$ ergibt sich die differentielle Form zu

\begin{align}
     \frac{\partial \rho}{\partial t}  + \nabla(\rho \vec{u}) = 0 \overset{\rho = \mathrm{const.}}{\Rightarrow} \nabla \cdot \vec{u} &= 0
\end{align}
Für ein inkompressibles Fluid mit konstanter Dichte verschwindet zusätzlich die zeitliche Änderung $\partial_t \rho$.

\subsection{Navier-Stokes Gleichung}

Analog zu der Newtonschen Bewegungsleichung eines Teilchens betrachten wir in der Strömungsmechanik die Impulsgleichung für ein infinitesimales Fluidelement.
Die Geschwindigkeit ändert sich dabei nicht nur mit der Zeit sondern auch durch die Bewegung mit der Strömung. Bewegt sich das Fluidelement in einen
Bereich höherer Geschwindigkeit muss dies zu einer Erhöhung des eigenen Impulses führen.
Die zeitliche Änderung ist dann über die subtantielle Ableitung gegeben, die allgemein für einen Skalar $S$ gemäß

\begin{align}
    \frac{\Dif S}{\dif t} &:= \pdn[S]{t} + \vec{u}\vec{\nabla}S = \pdn[S]{t} + u\pdn[S]{x} + v\pdn[S]{y}+ w\pdn[S]{z}
\end{align}

definiert ist.
Weiterhin müssen wir die Kräfte betrachten die auf das Fluidelement wirken. Hier lässt sich zwischen in- und externe Kräften $F_{ext}$ unterscheiden.
Die internenen Kräfte sind gegeben durch den Druck und die innere Reibung.
Im folgendenen sei der Druck definiert als $p(\vec{r}) \in \mathbb{R}$. Bei vorliegen eines Druckgradientens wird sich ein Fluidelement in Richtung des niedrigen Druckes
also entgegen den Gradienten bewegen. Die Druckkraft ist daher gegeben als $-\nabla p$.
ÜBERARBEITEN
Bewegen sich Fluidelemente mit unterschiedlicher Geschwindigkeit zueinander so führt dies zu einer Scherspannung $\tau$, welche
für ein newtonsches Fluid linear von der Schergeschwindigkeit $\gamma$ abhängt. Für die Reibungskraft gilt daher  $\tau = \nu \nabla\vec{u}$.
ÜBERARBEITEN
Ergänzen wir die oben beschriebenen Terme in die Impulsgleichung so erhalten wir die Navier-Stoken gleichung
\begin{align}
    \pdn[u]{t} + \vec{u} \cdot \vec{\nabla} \vec{u} &= - \frac{1}{\rho} \nabla p + \nu \Delta \vec{u} + \vec{F}_{ext}
\end{align}
mit der kinematischen Viskosität $\nu = \mu /\rho$.
Zusammen mit der Kontinuitätsgleichung besteht nun ein überbestimmtes Gleichungssystem um die
zeitliche Entwicklung des Systems zu bestimmen.

\subsection{Anfangs- und Randbedingungen}
Für eine eindeutige Lösung ist es weiterhin nötig den Anfangszustand des Systems vorzugeben.
Bei der numerischen Lösung ist zu beachten, dass z.B. eine Instabilität eine gewisse Störung des Systems vorraussetzt.
Da numerische Fluktuationen sehr viel kleiner sind als im Experimen, kann es in diesem Fall sinnvoll sein
ein Zufallsrauschen in den einzelnen Variablen vorzugeben.
Weiterhin muss das Verhalten der Ränder des Systems eindeutig definiert sein.
Im Verlaufe der Arbeit werden die folgenden Ränder betrachtet
\begin{description}
    \item[No-Slip ()] Die Geschwindigkeit am Rand ist Null, $\vec{u} = (0, 0, 0)^T$.
    \item[No-Flux ()] Die Ableitung einer Größe am Rand ist Null z.B. $\pdn[p]{\vec{n}} = 0$.
    \item[Free-Slip ()] Die Geschwindigkeitskompontente senkrecht zur Wand verschwindet.
    \item[Peridodisch ()] Blablabla.
\end{description}
hierbei bezeichnet $R$ den Rand.

\subsection{Entdimensionalisierung}

-Enddimensionalisierung

\subsection{Method AK  und so zusammenfassung--}


Für die numerische Lösung der Gleichungen wird die Methode der künstlichen Kompressibiltät verwendet, welche in Abschitt ()
ausführlicher betrachtet wird. Der Vollständigkeit halber soll erwähnt werden, dass die Bewegungsgleichung






-poissongleichung
-methode künstliche kompressibilität !! <-- rausnehmen nur in numerik
a
-energiegleichung


\section{Rotierendes System}
Wir betrachten nun ein rotierendes System welches  sich relativ zu einem Inertialsystem um die Rotationsachse $\vec{\Omega}$ dreht.
Für zeitliche Ableitung im rotierenden System ergibt sich nach [] der Operator
\begin{align}
    \left\pdn[]{t} \right|_{Inert.} &= \left\pdn[]{\vec{t}}\right|_{Rot.} + \Omega \times
\end{align}
Damit erhalten wir für die Beschleunigung eines Punktes im rotierenden System
\begin{align}
    \pdn[\vec{u}]{t} = -\pdn[]{t^2}(\Omega \times \vec{r}) = -\underbrace{2\Omega \times \vec{u}}_{I} - \underbrace{\Omega \times (\Omega \times \vec{r}}_{II})
\end{align}

Die beiden Terme auf der rechten Seite entsprechen der Corioliskraft (I) und der Zentrifugalkraft (II). Während die Corioliskraft abhängig von der Geschwindigkeit ist
kann die Zentrifugalkraft auch in Form eines Potentials dargestellt werden []
\begin{align}
    \Omega \times (\Omega \times \vec{r}) = - \nabla(\frac{1}{2}\Omega^2\vec{r}^{'2}) = -\Phi
\end{align}
wobei $\vec{r}^'$ der Abstand zur Drehachse ist. Durch Substitution $p^* = p - \Phi$
erhalten wir die Navier-Stokes Gleichung für das rotierende System \footnote{Im folgenden wird meinen wir mit $p$ immer $p^*$ }.
\begin{align}
    \pdn[u]{t} + 2\Omega \times \vec{u} +  \vec{u} \cdot \vec{\nabla} \vec{u} &= - \frac{1}{\rho} \nabla p + \nu \Delta \vec{u} + \vec{F}_{ext}
\end{align}

-kenngrößen
-intertialwellen
-reflektion etc

\section{Rayleigh-Benard System}
-temperatur
-entdim
-instabilität
-bilder





