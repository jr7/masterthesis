\chapter{Theoretical Principles}

\section{Introduction}

Prior to the development of a numerical model, it is necessary to give an exact theoretical description of the
fluid systems, which are investigated in the context of this thesis.
Hence this section contains a brief overview of the derivation and the properties of the fundamental equations of motions.
For a more detailed description, the interested reader is referred to \citep{ferziger99} on which section \ref{theorie:eqm1} is based.\\
In the second part of this chapter, the equations will be extended for new types of fluid systems.
This includes the motion of fluid in a rotating frame of reference and the Rayleigh-B\'{e}nard system.
Furthermore the physical properties of these systems will be discussed.

\section{The Equations of Motion}\label{theorie:eqm1}

At any time we examine a viscous, newtonian and incompressible fluid. The equations of motion for such a fluid can be derived by considering the conversation of
mass and momentum inside a fixed control volume $\Omega \subset \mathbb{R}^3$.
Within the fluid the momentum at the position $\vec{r} = (x, y, z)^T$  is  characterized by the velocity $\vec{u}(\vec{r}, t) = (u, v, w)^T \in \mathbb{R}^3$,
meanwhile the mass distribution is given by the density distribution$\rho(\vec{r}) \in \mathbb{R}$.

\subsection{Mass Conservation}

Let $\partial V$ be the enclosing surface and $\vec{n}$ the normal vector to the control volume $\Omega$.
For any intensive property $\phi$ the reynolds transport theorem states that
\begin{align}
    \pdn[]{t} \int_{\Omega_M} \rho \phi \dif \Omega = \pdn[]{t}\int_{\Omega} \rho \phi \dif \Omega + \int_{\partial\Omega} \rho \phi \vec{v} \vec{n} \dif S
\end{align}
where $\Omega_M$ is a control mass (CM) volume, thus the time dependent volume of a fluid element passing through $\Omega$
\footnote{For one point in time it holds that $\Omega_M = \Omega$}.

By setting $\phi = 1$ one obtains the integral form of mass conversation.

\begin{align}
    \frac{\dif}{\dif t} \int_{\Omega_M}\dif V \rho(t) =  \int_{\Omega}\dif V \frac{\dif \rho}{\dif t}  &= \underbrace{-\int_{\partial \Omega}
     \rho \vec{u}\vec{n}\dif S}_{\mathrm{Massflux} \atop \mathrm{through \ } \partial \Omega} \stackrel{\text{Gauss} \atop \text{ Law}}{=}
      -\int_\Omega \dif V \vec{\nabla}\left(\rho \vec{u}\right)
\end{align}

The differential form of the equation is obtained by applying gauss law and considering an infinitesimal small control volume.
\begin{align}
     \frac{\partial \rho}{\partial t}  + \nabla \left(\rho \vec{u}\right) &= 0
\end{align}
As we investigate an incompressible fluid, which means $\rho = \text{const.}$, we get the incompressible continuity equation
\begin{align}
     \nabla \cdot \vec{u} &= 0
\end{align}

\subsection{Momentum Conservation}

Using the same approach, but with setting $\phi = \vec{v}$, results in the integral form of the momentum equation

\begin{align}
    \label{theorie:intimpulse}
    \pdn[]{t} \int_\Omega \rho \vec{u}\dif V + \int_{\partial\Omega} \rho\vec{u}\vec{u}\cdot \vec{n} \dif S =  \sum \vec{F}_{\text{ext.}} + \sum \vec{F}_{\text{int.}}
\end{align}

In addition to the left hand side, the equation is extended by additional internal and external forces, which may act on the fluid inside the control volume.
The external forces depend on the specific system  we examine, for example the buyont or the coriolis force, whereas the internal forces
are given by the pressure and the normal and shear stresses acting on a fluid element.\\
For a newtonian fluid the internal forces can be described by the stress tensor $\bm{T}$

\begin{align}
    \sum \vec{F}_{int.} = \int_{\partial \Omega} \bm{T}\vec{n} \dif S = \int_{\Omega} \dif V \nabla \bm{T} =
     \int_{\Omega} \dif V \nabla \left(- \left(p + \frac{2}{3}\mu\nabla\vec{u} \right) + 2\mu \bm{D} \right)
\end{align}
with the static pressure $p$, the dynamic viscosity $\mu$ and the deformation tensor $\bm{D}$.
Again we apply gauss law to equation \ref{theorie:intimpulse} and consider an infinitesimal volume.
The differential form of the impulse equation, also known as Navier-Stokes equation is than given by
\footnote{The term Navier-Stokes equation is generally referred to as the complete set of equations of motion or
just the impulse equation here we use the latter convention.}
\begin{align}
    \label{theorie:eqns}
    \pdn[u]{t} + \underbrace{\left(\vec{u} \vec{\nabla}\right) \vec{u}}_{\text{I}} &= \underbrace{- \frac{1}{\rho} \nabla p + \nu \Delta \vec{u}}_{\text{II}} +\sum \vec{F}_{\text{ext.}}
\end{align}
where we introduce the kinematic viscosity by the definition $\nu = \mu/\rho$.
For an incompressible fluid the force term generated by $\bm{T}$ reduces to (II).\\
Alltogether the internal force is now given by the pressure gradient and an diffusive impulse transport proportional to $\nu$.
The non-linearity of the fluid originates through term (I), which is also denoted as the advection operator.
It basically describes the change of impulse of a fluid element when moving through the velocity field.
\footnote{For example the velocity of a fluid element will change when no forces but a velocity field is present.}
It should be noted that the set of equations is not yet solvable as the pressure variable is still undetermined.

\subsection{Initial State and Boundary Conditions}

The solution of a partial differential equation, if it exists, is always undetermined by a constant of integration.
In order to determine the temporal evolution of a fluid system, it is necessary to define its initial state and therefore
to determine one specific solution.\\
This means that for every variable we have to choose an initial condition.
With respect to a numerical solution it has to be considered that, for example an instability, is always triggered by some kind of disturbance.
Thus it might be advisable to not choose a trivial solution like a zero velocity field, but instead a solution which quickly evolves in the state
of the system one wants to achieve. Many times it is useful to add pseudorandom noise.\\
\\
Since the fluid domain is spatial restricted it is as well necessary to define the physical behavior at its boundaries.
For a fluid domain $\Omega$ with the boundary $\partial \Omega$, the following boundarys are considered, as defined in \citep{Griebel1998} and \citep{ferziger99}.

\begin{description}
    \item[No-Slip Boundaries] All velocities components are set to zero $\vec{v}|_{\partial \Omega} = 0$. The fluid is at rest on $\partial \Omega$ and no flux through
                              the boundary occurs.
                              In a more general case, this kind of condition is  also referred to as Dirichlet-Condition where $\Phi|_{\partial \Omega} = c\in\mathbb{R} $,
                              for any variable $\Phi$.

    \item[Free-Slip Boundaries] The velocity component in normal direction to the wall is set to zero, hence $\vec{n} \nabla \vec{v} = 0$ and $\vec{n}\vec{v}=0$ is required.
                                No flux through $\partial \Omega$ occurs and no friction is impossed on the fluid.

    \item[No-Flux Boundaries] For a scalar $\Phi$, the flux through the boundary is zero, hence $\vec{n}\nabla \Phi = 0$.
                              Here the general case is referred to as Neumann boundary condition, where $\vec{n}\nabla \Phi = c\in \mathbb{R}$.
                              This boundary conditions is used i.e. to avoid energy flux through the domain boundary.

    \item[Periodic boundaries] Periodic boundaries can be applied in all directions of a system. For example if the system is periodic in x-Direction with a period $L$,
                                one has to ensure that all variables match on the boundaries, that is $\Phi(x) = \Phi(x + L)$, for any variable $\Phi$.
\end{description}

\subsection{Nondimensionalization}

For many fluid systems nondimensionalization is used to further symplify the equations of motion and reduce the number of free parameters.
The approach behind this scheme is to define variable substitutions, such that the overall systems is free from any physical units.\\
After the nondimensionalization the system is described by one or more dimensionless quantity, which characterize the overall physical behaviour.
Therefore it is easier to compare numerical simulations and experimental setups to one another.
We can choose the following scales for the variables of time, position, velocity and pressure (see \cite{Kundu2012}), the nondimensional variables are denoted by an asterisk.

\begin{align}
    \text{Length:}\qquad &  \vec{x}^* = \frac{\vec{x}}{L}  & \qquad \text{Velocity:}\qquad& \vec{x}^* = \frac{\vec{v}}{V}\\
    \text{Time:}  \qquad & t^* = t \cdot \frac{V}{L}      & \qquad  \text{Pressure:}\qquad & p^* = \frac{p - p_\infty}{\rho V^2}
\end{align}

Here we choose $L$ as a length and $V$ for the velocity, as typical scales from the fluid system, we consider.
The pressure scale is set as a difference with respect to a characteristic pressure $p_\infty$.
With the above defined scales the nondimensionalized Navier-Stokes equation is given by \footnote{
From now on we ignore the * for all dimensionless variables }:

\begin{align}
    \pdn[u]{t} + \vec{u} \cdot \vec{\nabla} \vec{u} &= -\nabla p + \frac{1}{Re} \Delta \vec{u} + \vec{F}_{ext.}
\end{align}

The dynamic of the system is reduced to the dimensionless quantity $\Rey$, formerly known as the Reynolds number,
defined by (\citep{Kundu2012}):

\begin{align}
    \label{theorie:renumber}
    Re := \frac{VL}{\nu} = \frac{\rho VL}{\mu}
\end{align}

\newpage

From equation \ref{theorie:renumber}, it can be seen that the Reynolds number gives the ratio between the inertial forces $\propto VL$
to the viscous forces $\propto \nu$, of the fluid system.
This means that for a small $\Rey$ the viscous forces dominate, whereas for a large $\Rey$ we can expect advection driven, maybe even turbulent flow.\\
To illustrate the behaviour, figure \ref{theorie:re_example} exemplarly shows the airflow around a cylinder at different reynolds numbers.
TODO:\\
- bild\\
- comparison\\

\begin{figure}[!pb]
    \label{theorie:re_example}
  \centering
    \missingfigure[figwidth=\textwidth]{Flow over a cylinder for different Reynolds numbers}
\end{figure}
\newpage

\section{Rotational Fluid Dynamics}

In this section we want to extend the Navier-Stokes equations to govern the physical attributes of rotating systems,
which play an important role in the context of geophysical fluid dynamics.\\
Due to the continous acceleration acting on the fluid, these systems exhibit some fundamental different behavior, than one would expect.
One important case we want to discuss, is the internal propagation of inertial waves inside a stratisfied rotating fluid.\\

\subsection{Equations of Motion}

We initially consider the motion of fluid in a coordinate system (\textbf{R}), rotating relative to the intertial system (\textbf{I}) around the axis $\vec{\Omega}$.
The relation of the time derivate between the two frames of motion is given by the relation
\begin{align}
    \left.{\pdn[]{t}}\right|_{\bm{I}} = \left.\pdn[]{t}\right|_{\bm{R}} + \Omega \times
\end{align}
according to \citep{Tilgner2007}.
Applying this relation to the position vector relative to the coordinate systems \textbf{R} and \textbf{I}
and by furthermore assuming a constant rotation rate, thus $\partial_t\vec{\Omega} = 0$, yields a coordinate transformation for the acceleration.

\begin{align}
    \label{theorie:rottrafo}
    \left.\pdn[\vec{u}]{t}\right|_{\bm{I}} = \left(\left.\pdn[\vec{u}]{t}  + \underbrace{2\vec{\Omega} \times \vec{u}|_R}_{\text{I}}
    - \underbrace{\vec{\Omega} \times (\vec{\Omega} \times \vec{r|_R}}_{\text{II}})\right)\right|_{\bm{R}}
\end{align}

Hence, the transition into a rotating coordinate system introduces two additional translucent forces, the coriolis force (I)  and the centrifugal force (II).\\
By substituting expression \ref{theorie:rottrafo} into the Navier-Stokes equation, we obtain the equations of motion for the rotating system.
A further simplication can be obtained by considering that the centrifugal force is independent of the velocity field,
hence it can be written in terms of a potential $\Phi$

\begin{align}
    \Omega \times (\Omega \times \vec{r}) = - \nabla(\frac{1}{2}\Omega^2\vec{r}^{'2}) = -\Phi
\end{align}

which can be substituted into the pressure gradient by defining $p^* = p - \Phi$ \citep{tritton88}.
\newpage
Last but not least we choose the following scales to obtain a nondimensional equation

\begin{align}
    \text{Length:}\qquad &  \vec{x}^* = \frac{\vec{x}}{L}  &
    \qquad \text{Velocity:}\qquad& \vec{u}^* =  \frac{\vec{u}}{|\vec{\Omega}|L}\\
    \text{Time:}  \qquad & t^* = t \cdot |\vec{\Omega}| &
    \qquad  \text{Pressure:}\qquad & p^* = \frac{p - p_\infty}{\rho L^2{|\vec{\Omega}|}^2}
\end{align}

The final form of the Navier-Stokes equation for the rotating coordinate system reads

\begin{align}
    \pdn[u]{t}+ \left(\vec{u}  \vec{\nabla}\right) \vec{u} + 2\Omega \times \vec{u}  &= -  \nabla p + \Ekman \Delta \vec{u} + \vec{F}_{\text{ext.}}
\end{align}

with the the dimensionless quantity $\Ekman$ the Ekman-Number, which is defined by

\begin{align}
    \Ekman := \frac{\nu}{|\vec{\Omega}| L^2}
\end{align}

and describes the ratio between the Coriolis and viscous forces inside the fluid.

\subsection{Inertial Waves}

In rotating systems one possible solution is the propagation of so called inertial waves, through the fluid.

Let us recall that 1



-what is a wave definition
-stratisfaction drehimpuls
-auslenkung coriolis force

-we begin with

For a better unterstanding, we want to make a short comparison to internal gravity waves.
Suppose we
-sattelite myabe

In a mathematical sence the fundamental properties of an inertial wave is given by the dispersion relation, the group and phase velocity.
To obtain these expressions, the common approach is the assumption of solutions of the type $e^{i(k_j x_j - \omega t)}$.
By inserting the ansatz into equation () one obtains the solution




-what are gravitational waves
- restronig force eqilibrium

- drehimpuls





-einleitung - blabla mit stratisfaction drehimpuls

-lösung der gleichungen  bla bla bla

-eigen schaften von intertial waves

-reflektion

\newpage
\section{Rayleigh-Benard System}
Not important yet.... \\
-temperatur\\
-entdim\\
-instabilität\\
-bilder\\





