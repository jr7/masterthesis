\chapter{Theoretische Grundlagen}

\section{Einleitung}

Eine numerische Modellierung setzt zunächst eine exakte theoretische Beschreibung des strömungsmechanischen Problems vorraus.
Daher soll in diesem Abschnitt auf die theoretischen Grundlagen der Strömungsmechanik eines inkompressiblen Fluids eingegangen werden.
Auf eine exakte mathematische Herleitung wird dabei verzichtet, der interessierte Leser sei auf [Ferziger?] oder z.B. [rritton?] verwiesen.
Anschließend werden die Grundgleichungen für verschiedene Systeme erweitert, dazu gehört die Strömung in rotierenden Systemen sowie das Rayleigh-Benard-System.

\section{Die Navier-Stokes Gleichung}

Für die Charakterisierung einer Strömung an einem Punkt $\vec{r} = (x, y, z)^T$  im Gebiet $\Omega \subset \mathbb{R}^3$ betrachten wir zunächst
die Geschwindigkeit $\vec{u}(\vec{r}, t) = (u, v, w)^T \in \mathbb{R}^3$ und die Dichteverteilung $\rho(\vec{r}) \in \mathbb{R}$.

\subsection{Kontinuitätsgleichung}

Sei nun $V \subset \Omega$  ein abschlossenes Teilvolumen mit einem zu seiner Oberfläche $\partial V$ gegebenen normalen Vektor $\dif \vec{S}$.
Infolge der Massenerhaltung können wir annehmen, dass sich die Gesamtmasse innerhalb von V nur ändert wenn ein
Transport $\propto \vec{u}\dif \vec{S}$ in- oder aus dem Volumen stattfindet.
Mathematisch präziser formuliert und mit Verwendung des Satz von Gauß (*) folgt die integrale Form der Kontinuitätsgleichung.

\begin{align}
    \frac{\dif}{\dif t} \int_{V}\dif V \rho(t) =  \int_{V}\dif V \frac{\dif \rho}{\dif t}  &= \underbrace{-\int_{\partial V}
     \rho \vec{u}\dif \vec{S}}_{\mathrm{Massenfluss} \atop \mathrm{durch \ } \partial V} \overset{*}{=} -\int_V \dif V \vec{\nabla}\vec{u}
\end{align}

Durch Bildung des Limes $V\rightarrow 0$ ergibt sich die differentielle Form zu

\begin{align}
     \frac{\partial \rho}{\partial t}  + \nabla(\rho \vec{u}) = 0 \overset{\rho = \mathrm{const.}}{\Rightarrow} \nabla \cdot \vec{u} &= 0
\end{align}
Für ein inkompressibles Fluid mit konstanter Dichte verschwindet zusätzlich die zeitliche Änderung $\partial_t \rho$.

\subsection{Impulsgleichung}

Analog zu der Newtonschen Bewegungsleichung eines Teilchens betrachten wir in der Strömungsmechanik die Impulsgleichung für ein infinitesimales Fluidelement.
Die Geschwindigkeit ändert sich dabei nicht nur mit der Zeit sondern auch durch die Bewegung mit der Strömung. Bewegt sich das Fluidelement in einen
Bereich höherer Geschwindigkeit muss dies zu einer Erhöhung des eigenen Impulses führen.
Die zeitliche Änderung ist dann über die subtantielle Ableitung gegeben, die allgemein für ein Skalar $S$ gemäß

\begin{align}
    \frac{\Dif S}{\dif t} &:= \pdn[S]{t} + \vec{u}\vec{\nabla}S = \pdn[S]{t} + u\pdn[S]{x} + v\pdn[S]{y}+ w\pdn[S]{z}
\end{align}

definiert ist.
Weiterhin müssen wir die Kräfte betrachten die auf das Fluidelement wirken. Hier lässt sich zwischen in- und externe Kräften $F_{ext}$ unterscheiden.
Die internenen Kräfte sind gegeben durch die Druckkraft und die Viskosität.

-Druck
-Viskostität Newton gleichung

Damit lautet die Navier-Stokes Gleichung für ein inkompressibles Fluid.
\begin{align}
    \pdn[u]{t} + \vec{u} \cdot \vec{\nabla} \vec{u} &= - \frac{1}{\rho} \nabla p + \nu \Delta \vec{u} + \vec{F}_{ext}\\
\end{align}

-Randbedingungen
-Enddimensionalisierung

\section{Rotierendes System}
Wir betrachten nun ein rotierendes System welches  sich relativ zu einem Inertialsystem um die Rotationsachse $\vec{\Omega}$ dreht. Nach [] gilt
\begin{align}
    \left\pdn[\vec{r}]{t} \right|_{Inert.} &= \left\pdn[\vec{r}]{\vec{t}}\right|_{Rot.} + \Omega \times \vec{r}
\end{align}
Damit folgt für die Beschleunigung eines Punktes im rotierenden System
\begin{align}
    \pdn[u]{t} = -\pdn[]{t^2}(\Omega \times \vec{r}) = -2\Omega \times \vec{u} - \Omega \times (\Omega \times \vec{r})
\end{align}

-Coriolis Kraft Herleiten
-intertialwellen
-reflektion etc

\section{Rayleigh-Benard System}
-temperatur
-entdim
-instabilität
-bilder





