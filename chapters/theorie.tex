\chapter{Theoretical Principles}

\section{Introduction}

Prior to the development of a numerical model, it is necessary to give an exact theoretical description of the
fluid systems, which are investigated in the context of this thesis.
Hence this section contains a brief overview of the derivation and the properties of the fundamental equations of motions.
For a more detailed description, the interested reader is referred to [], [] and [], on which section () is primary based.\\
In the second part of this chapter, the equations will be extended for new types of fluid systems.
This includes the motion of fluid in a rotating frame of reference and the Rayleigh-B\'{e}nard system.
Furthermore the physical properties of these systems will be discussed.

\section{The Equations of Motion}

At any time we examine a viscous, newtonian and incompressible fluid. The equations of motion for such a fluid can be derived by considering the conversation of
mass and momentum inside a fixed control volume $\Omega \subset \mathbb{R}^3$.
Within the fluid the momentum at the position $\vec{r} = (x, y, z)^T$  is  characterized by the velocity $\vec{u}(\vec{r}, t) = (u, v, w)^T \in \mathbb{R}^3$,
meanwhile the mass distribution is given by the density distribution$\rho(\vec{r}) \in \mathbb{R}$.

\subsection{Mass Conservation}

Let $\partial V$ be the enclosing surface and $\vec{n}$ the normal vector to the control volume $\Omega$.
For any intesive property $\phi$ the reynolds transport theorem states that
\begin{align}
    \pdn[]{t} \int_{\Omega_M} \rho \phi \dif \Omega = \pdn[]{t}\int_{\Omega} \rho \phi \dif \Omega + \int_{\partial\Omega} \rho \Phi \vec{v} \vec{n} \dif S
\end{align}
where $\Omega_M$ is the volume of our control mass (EXPLANATION? Partial /diff?).
By setting $\phi = 1$ one obtains the integral form of mass conversation.

\begin{align}
    \frac{\dif}{\dif t} \int_{V}\dif V \rho(t) =  \int_{V}\dif V \frac{\dif \rho}{\dif t}  &= \underbrace{-\int_{\partial V}
     \rho \vec{u}\dif \vec{S}}_{\mathrm{Massenfluss} \atop \mathrm{durch \ } \partial V} \overset{*}{=} -\int_V \dif V \vec{\nabla}\vec{u}
\end{align}

In a physical sense this means a change of mass inside $\Omega$ can only occur by a mass transport trough $\partial \Omega$ or by a change in the density.
The differential form of the equation is obtained by considering an infinitesimal small control volume and applying gauss law

\begin{align}
     \frac{\partial \rho}{\partial t}  + \nabla(\rho \vec{u}) = 0 \overset{\rho = \mathrm{const.}}{\Rightarrow} \nabla \cdot \vec{u} &= 0
\end{align}

As we investigate an incompressible fluid, which means $\rho = const.$, we get the incompressible continuity equation
\begin{align}
     \frac{\partial \rho}{\partial t}  + \nabla(\rho \vec{u}) = 0 \overset{\rho = \mathrm{const.}}{\Rightarrow} \nabla \cdot \vec{u} &= 0
\end{align}


\subsection{Momentum Conservation}

Using the same approach, but with setting $\phi = \vec{v}$, results in the momentum equation
\begin{align}
    \pdn[\rho \vec{u}]{t} + \nabla\left(\rho\vec{u}\vec{u}\right) =  -\nabla p  + \vec{f}_{int} + \vec{f}_{ext}
\end{align}
In addition the right hand side of the equation consist of the internal and external forces which act on the fluid.
The external forces depend on the specific system  we examine i.e. buyont or the coriolis force, whereas the internal forces
are given by the pressure and the normal and shear stresses of the fluid.
The pressure forces can be derived as $- \nabla p$.
The forces induced by normal and shear stresses can be derived by the divergence of the stress tensor $T$.
For an incompressible and newtonian fluid the expression simplifies to $\nu \Delta \vec{u}$ where $\nu$ is the kinematic viscosity of the fluid.
Inserting these expression into the momentum equation results in the navier stokes equation:

\begin{align}
    \pdn[u]{t} + \vec{u} \cdot \vec{\nabla} \vec{u} &= - \frac{1}{\rho} \nabla p + \nu \Delta \vec{u} + \vec{F}_{ext}
\end{align}


\subsection{Initial and Boundary Conditions}

To determine the temporal evolution of the given equations it is necessary to define an initial set of conditions for all variables.
With respect to a numerical solution it has to be considered that, for example an instability, is always triggered by some kind of disturbance.
Thus it might be advisable to not choose a trivial solution like a zero velocity field, but instead a solution which quickly evolves in the state
of the system one wants to achieve. Many times it is useful to add some pseudo random noise to the initial state.\\
Since the fluid domain is spatial restricted it is as well necessary to define the physical behavior at its boundaries.
In the thesis the following boundary conditions are considered


\begin{description}
    \item[No-Slip Boundaries] The value of all velocities component is set to zero $\vec{v}|_{\partial \Omega} = 0$.
                              More generally speaking, this kind of condition is  also referred to as Dirichlet-Condition where $\Phi=c $, for a scalar field bla [].
    \item[Free-Slip Boundaries] The velocity component in normal direction to the wall is set to zero, this is fulfilled with $\vec{n}\pdn[v]{\vec{n}} = 0$.
                                Free-Slip boundaries are used to reduce wall friction or to mimic symmetric bla.
    \item[No-Flux Boundaries] For a scalar $\phi$ the flux trough the wall is zero which means $\pdn[\phi]{\vec{n}}$.
                              Here the general case is referred to as VonNeumann-Condition, where $bla=bla$, for a scalalr $bla$  [].
                              This boundary conditions is used to avoid mass or energy flux trough the domain boundary.
    \item[Periodic boundaries] Periodic boundaries can be applied in all directions of our system. For example if the system is periodic in x-Direction,
                                one has to ensure that $\Phi(x) = \Phi(x + L)$, where $x \in L$.

\end{description}

The excact implementation of these methods is dicussed in the numerical section.

\subsection{Nondimensionalization}

Enddimensionalisierung

\subsection{Method AK  und so zusammenfassung}


\section{Rotierendes System}
Wir betrachten nun ein rotierendes System welches  sich relativ zu einem Inertialsystem um die Rotationsachse $\vec{\Omega}$ dreht.
Für zeitliche Ableitung im rotierenden System ergibt sich nach [] der Operator


\iffalse
\begin{align}
    \left\pdn{t} \right|_{Inertial} &= \left\pdn[]{\vec{t}}\right|_{Rotated} + \Omega \times
\end{align}
\fi
Damit erhalten wir für die Beschleunigung eines Punktes im rotierenden System
\begin{align}
    \pdn[\vec{u}]{t} = -\pdn[]{t^2}(\Omega \times \vec{r}) = -\underbrace{2\Omega \times \vec{u}}_{I} - \underbrace{\Omega \times (\Omega \times \vec{r}}_{II})
\end{align}

Die beiden Terme auf der rechten Seite entsprechen der Corioliskraft (I) und der Zentrifugalkraft (II). Während die Corioliskraft abhängig von der Geschwindigkeit ist
kann die Zentrifugalkraft auch in Form eines Potentials dargestellt werden []
\begin{align}
    \Omega \times (\Omega \times \vec{r}) = - \nabla(\frac{1}{2}\Omega^2\vec{r}^{'2}) = -\Phi
\end{align}
wobei $\vec{r}^s$ der Abstand zur Drehachse ist. Durch Substitution $p^* = p - \Phi$
erhalten wir die Navier-Stokes Gleichung für das rotierende System \footnote{Im folgenden wird meinen wir mit $p$ immer $p^*$ }.
\begin{align}
    \pdn[u]{t} + 2\Omega \times \vec{u} +  \vec{u} \cdot \vec{\nabla} \vec{u} &= - \frac{1}{\rho} \nabla p + \nu \Delta \vec{u} + \vec{F}_{ext}
\end{align}

-kenngrößen
-intertialwellen
-reflektion etc

\section{Rayleigh-Benard System}
-temperatur
-entdim
-instabilität
-bilder





