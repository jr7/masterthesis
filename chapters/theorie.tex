\chapter{Theoretical Principles}

\section{Introduction}

Prior to the development of a numerical model, it is necessary to give an exact theoretical description of the
fluid systems, which are investigated in the context of this thesis.
Hence this section contains a brief overview of the derivation and the properties of the fundamental equations of motions.
For a more detailed description, the interested reader is referred to [], [] and [], on which section () is primary based.\\
In the second part of this chapter, the equations will be extended for new types of fluid systems.
This includes the motion of fluid in a rotating frame of reference and the Rayleigh-B\'{e}nard system.
Furthermore the physical properties of these systems will be discussed.

\section{The Equations of Motion}

At any time we examine a viscous, newtonian and incompressible fluid. The equations of motion for such a fluid can be derived by considering the conversation of
mass and momentum inside a fixed control volume $\Omega \subset \mathbb{R}^3$.
Within the fluid the momentum at the position $\vec{r} = (x, y, z)^T$  is  characterized by the velocity $\vec{u}(\vec{r}, t) = (u, v, w)^T \in \mathbb{R}^3$,
meanwhile the mass distribution is given by the density distribution$\rho(\vec{r}) \in \mathbb{R}$.

\subsection{Mass Conservation}

Let $\partial V$ be the enclosing surface and $\vec{n}$ the normal vector to the control volume $\Omega$.
For any intesive property $\phi$ the reynolds transport theorem states that
\begin{align}
    \pdn[]{t} \int_{\Omega_M} \rho \phi \dif \Omega = \pdn[]{t}\int_{\Omega} \rho \phi \dif \Omega + \int_{\partial\Omega} \rho \Phi \vec{v} \vec{n} \dif S
\end{align}
where $\Omega_M$ is the volume of our control mass (EXPLANATION? Partial /diff?).
By setting $\phi = 1$ one obtains the integral form of mass conversation.

\begin{align}
    \frac{\dif}{\dif t} \int_{V}\dif V \rho(t) =  \int_{V}\dif V \frac{\dif \rho}{\dif t}  &= \underbrace{-\int_{\partial V}
     \rho \vec{u}\dif \vec{S}}_{\mathrm{Massenfluss} \atop \mathrm{durch \ } \partial V} \overset{*}{=} -\int_V \dif V \vec{\nabla}\vec{u}
\end{align}

In a physical sense this means a change of mass inside $\Omega$ can only occur by a mass transport trough $\partial \Omega$ or by a change in the density.
The differential form of the equation is obtained by considering an infinitesimal small control volume and applying gauss law

\begin{align}
     \frac{\partial \rho}{\partial t}  + \nabla(\rho \vec{u}) = 0 \overset{\rho = \mathrm{const.}}{\Rightarrow} \nabla \cdot \vec{u} &= 0
\end{align}

As we investigate an incompressible fluid, which means $\rho = const.$, we get the incompressible continuity equation
\begin{align}
     \frac{\partial \rho}{\partial t}  + \nabla(\rho \vec{u}) = 0 \overset{\rho = \mathrm{const.}}{\Rightarrow} \nabla \cdot \vec{u} &= 0
\end{align}


\subsection{Momentum Conservation}

Using the same approach, but with setting $\phi = \vec{v}$, results in the momentum equation
\begin{align}
    \pdn[\rho \vec{u}]{t} + \nabla\left(\rho\vec{u}\vec{u}\right) =  -\nabla p  + \vec{f}_{int} + \vec{f}_{ext}
\end{align}
In addition the right hand side of the equation consist of the internal and external forces which act on the fluid.
The external forces depend on the specific system  we examine i.e. buyont or the coriolis force, whereas the internal forces
are given by the pressure and the normal and shear stresses of the fluid.
The pressure forces can be derived as $- \nabla p$.
The forces induced by normal and shear stresses can be derived by the divergence of the stress tensor $T$.
For an incompressible and newtonian fluid the expression simplifies to $\nu \Delta \vec{u}$ where $\nu$ is the kinematic viscosity of the fluid.
Inserting these expression into the momentum equation results in the navier stokes equation:

\begin{align}
    \pdn[u]{t} + \vec{u} \cdot \vec{\nabla} \vec{u} &= - \frac{1}{\rho} \nabla p + \nu \Delta \vec{u} + \vec{F}_{ext}
\end{align}


\subsection{Initial and Boundary Conditions}

To determine the temporal evolution of the given equations it is necessary to define an initial set of conditions for all variables.
With respect to a numerical solution it has to be considered that, for example an instability, is always triggered by some kind of disturbance.
Thus it might be advisable to not choose a trivial solution like a zero velocity field, but instead a solution which quickly evolves in the state
of the system one wants to achieve. Many times it is useful to add some pseudo random noise to the initial state.\\
Since the fluid domain is spatial restricted it is as well necessary to define the physical behavior at its boundaries.
In the thesis the following boundary conditions are considered


\begin{description}
    \item[No-Slip Boundaries] The value of all velocities component is set to zero $\vec{v}|_{\partial \Omega} = 0$.
                              More generally speaking, this kind of condition is  also referred to as Dirichlet-Condition where $\Phi=c $, for a scalar field bla [].
    \item[Free-Slip Boundaries] The velocity component in normal direction to the wall is set to zero, this is fulfilled with $\vec{n}\pdn[v]{\vec{n}} = 0$.
                                Free-Slip boundaries are used to reduce wall friction or to mimic symmetric bla.
    \item[No-Flux Boundaries] For a scalar $\phi$ the flux trough the wall is zero which means $\pdn[\phi]{\vec{n}}$.
                              Here the general case is referred to as VonNeumann-Condition, where $bla=bla$, for a scalalr $bla$  [].
                              This boundary conditions is used to avoid mass or energy flux trough the domain boundary.
    \item[Periodic boundaries] Periodic boundaries can be applied in all directions of our system. For example if the system is periodic in x-Direction,
                                one has to ensure that $\Phi(x) = \Phi(x + L)$, where $x \in L$.

\end{description}

The excact implementation of these methods is dicussed in the numerical section.

\subsection{Nondimensionalization}

For many fluid systems nondimensionalization is used to further symplify the equations of motion and reduce the number of free parameters.
The approach behind this scheme is to define variable substitutions, such that the overall systems is free from any physical units.
After the nondimensionalization the system is described by one or more dimensionless quantity, which characterize the overall physical behaviour.
Therefore the nondimensionalization makes it easier to compare numerical simulations and experimental setups to one another.\\
For the equations () and (), we can choose the following scales (see []).

\begin{align}
    \mathrm{Length}  : \vec{x}^* = \frac{\vec{x}}{L}  &, \mathrm{Velocity}: \vec{x}^* = \frac{\vec{v}}{V}\\
    \mathrm{Time}    : t^* = t \cdot \frac{V}{L}      &, \mathrm{Pressure}: p^* = \frac{p}{\rho V^2}
\end{align}

Here we choose $L$ as a lenght scale of our system and $V$ is the velocity scale of our system.
For example in a pipe flow experiment we would choose $L$ as the diameter of our pipe and $V$ by the maximum velocity observed.
The time can simply be scaled  by $V/L$ , furthemore for the pressure there is no natural lbab3lab3ba.
With the above defined nondimensionalizton the impulse equation reduced to \footnote{Here we ignored *}

\begin{align}
    \pdn[u^*]{t^*} + \vec{u^*} \cdot \vec{\nabla} \vec{u^*} &= -\nabla p^* + \frac{1}{Re} \Delta \vec{u^*} + \vec{F^*}_{ext}
\end{align}

where $Re$ is now defined as the Reynolds-number given by [cite].
\begin{align}
    Re = \frac{VL}{\nu} = \frac{\rho VL}{\mu} = \frac{\mathrm{intertial.forces}}{\mathrm{viscous.forces}}
\end{align}

The system is now charaterized by one dimensionless quantity, the Reynolds-number, which describes the ratio of inertial to the viscous forces.
As exemplarly shown in image (),  for a small reynolds number (<soundso\footnote{system dependent})  the viscous forces are dominating and we can expect laminar flow regimes, whereas for
large Reynolds number the flow becomes turbulent.

\newpage

\section{Fluid Flow in Rotating Systems}
\subsection{Equations of Motion}

We now consider the motion of fluid in a coordinate system (\textbf{R}), rotating relative to an intertial system (\textbf{I}) around the axis $\vec{\Omega}$.
The relation of the time derivate, between the two frames of motion, is according to [], given by $(\partial/\partial t)_I = (\partial/\partial t)_R + \Omega \times $ [cite].
Applying this relation to the position vector $\vec{r}_R$ and  $\vec{r}_I$, with the assumption that $\Omega=\mathrm{const'}$, gives the expression

\begin{align}
    \left.\pdn[\vec{u}]{t}\right|_I r= \left.\pdn[\vec{u}]{t}\right|_R   + \underbrace{2\Omega \times \vec{u}|_R}_{R} - \underbrace{\Omega \times (\Omega \times \vec{r|_R}}_{II})
\end{align}

Hence, the transition into the rotating frame generates two additional forces, the coriolis force (I)  and the centrifugal force (II).
By inserting this expression into the impulse equation we obtain the equations of motion for the rotating system.
Furthermore, since the centrifugal force is independent of the velocity field it can be written in terms of a potential $\Phi$

\begin{align}
    \Omega \times (\Omega \times \vec{r}) = - \nabla(\frac{1}{2}\Omega^2\vec{r}^{'2}) = -\Phi
\end{align}

which  can than be substituted into the pressure by setting $p^* = p - \Phi$.
This simplifies the equation to

\begin{align}
    \pdn[u]{t} + 2\Omega \times \vec{u} +  \vec{u} \cdot \vec{\nabla} \vec{u} &= - \frac{1}{\rho} \nabla p + \nu \Delta \vec{u} + \vec{F}_{ext}
\end{align}

for the nondimensionalization we use scales proportional to the rotation axis $\Omega$.

\begin{align}
    \mathrm{Length}  : \vec{x}^* = \frac{\vec{x}}{L}  &, \mathrm{Velocity}: \vec{u}^* = \frac{\vec{u}}{\Omega L}\\
    \mathrm{Time}    : t^* = t \cdot \Omega      &, \mathrm{Pressure}: p^* = \frac{p}{\Omega^2 L^2}
\end{align}

The dimensionless equation reads

\begin{align}
    \pdn[u]{t} + 2\Omega \times \vec{u} +  \vec{u} \cdot \vec{\nabla} \vec{u} &= - \frac{1}{\rho} \nabla p + \nu \Delta \vec{u} + \vec{F}_{ext}
\end{align}

with the the dimensionless quantity $Ek$ the Ekman-Number, which is given by

\begin{align}
    Ek = \frac{\nu}{\Omega L^2}
\end{align}

and describes the ratio between the Coriolis and viscous forces inside the fluid.

\subsection{Inertial Waves}

In rotating systems one possible solution is the propagation of so called inertial waves, through the fluid.

For a better unterstanding, we want to make a short comparison to internal gravity waves.
Suppose we
-sattelite myabe

In a mathematical sence the fundamental properties of an inertial wave is given by the dispersion relation, the group and phase velocity.
To obtain these expressions, the common approach is the assumption of solutions of the type $e^{i(k_j x_j - \omega t)}$.
By inserting the ansatz into equation () one obtains the solution




-what are gravitational waves
- restronig force eqilibrium

- drehimpuls





-einleitung - blabla mit stratisfaction drehimpuls

-lösung der gleichungen  bla bla bla

-eigen schaften von intertial waves

-reflektion

\section{Rayleigh-Benard System}
-temperatur
-entdim
-instabilität
-bilder





