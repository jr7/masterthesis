\chapter{Theoretische Grundlagen}

\section{Einleitung}

Eine numerische Modellierung setzt zunächst eine exakte theoretische Beschreibung des strömungsmechanischen Problems vorraus.
Daher soll in diesem Abschnitt auf die theoretischen Grundlagen der Strömungsmechanik eines inkompressiblen Fluids eingegangen werden.
Auf eine exakte mathematische Herleitung wird dabei verzichtet, der interessierte Leser sei auf [Ferziger?] oder z.B. [rritton?] verwiesen.
Anschließend werden die Grundgleichungen für verschiedene Systeme erweitert, dazu gehört die Strömung in rotierenden Systemen sowie das Rayleigh-Benard-System.

\section{Die Navier-Stokes Gleichung}

Für die Charakterisierung einer Strömung in einem Gebiet $\Omega \in \mathbb{R}^3$ werden die folgenden Skalare betrachtet.
\begin{description}
    \item[$\bm{\vec{v}_x}$, $\bm{\vec{v}_y}$, $\bm{\vec{v}_z}$] Geschwindigkeit in x, y, z Richtung
    \item[$\bm{\rho}$] Dichte
    \item[$\bm{p}$] Druck
\end{description}

Betrachten wir zunächst ein abschlossenes Teilvolumen $V \in \Omega$ mit einem normalen Vektor $\dif \vec{S}$.
So ergibt sich die Massenbilanz aus der zeitlichen Änderung der Dichte und dem Massenfluss aus dem Volumen.

\begin{align}
    \frac{\dif m}{\dif t} &= \frac{\dif}{\dif t} \int_{V}\dif V \rho(t)  = -\int_{\partial V} \rho \vec{v}\dif \vec{s} \\
\end{align}

Unter der Annahme der Massenerhaltung und der Verwendung des Gaußschen Integralsatz folgt
die Kontinuitätsgleichung

\begin{align}
     \int_{V} \left( \frac{\partial \rho}{\partial t}  + \nabla(\rho \vec{u}) \right) = 0 \Rightarrow \frac{\partial \rho}{\partial t}  + \nabla(\rho \vec{u}) \\
\end{align}

für inkompressible Fluide $\rho = \rho_0}$ vereinfacht sich die Gleichung zu

\begin{align}
    \nabla \cdot \vec{u} &= 0\\
\end{align}

Im analogon zur Newtonschen Bewegungsleichung eines Teilchens betrachten wir in der Strömungsmechanik die Impulsgleichung für ein infinitesimales Fluidelement.
Der Impuls ändert sich in Abhängigkeit von der Zeit und durch die Bewegung des Fluidelements durch das Geschwindigkeitsfeld. Das Verhalten wird durch die substantielle Ableitung beschrieben.
UMSCHREIBEN OBEN FALSCH U UND V TAUSCHEN AUF UU

\begin{align}
    \pdn[f]{t} &= \pdn[f]{t} + \vec{u}\vec{\nabla}f = \pdn[f]{t} + u\pdn[f]{x} v\pdn[f]{y}+ w\pdn[f]{z} \\
\end{align}


- erläuterung friction tensor
- f ext

Damit lautet die Navier-Stokes Gleichung für ein inkompressibles Fluid.
\begin{align}
    \pdn[u]{t} + \vec{u} \cdot \vec{\nabla} \vec{u} &= - \frac{1}{\rho} \nabla p + \nu LAPLACE \vec{u} + \vec{F}_{ext}\\
\end{align}



