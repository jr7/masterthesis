\chapter{Python Simulation API}

\section{Introduction}

In this chapter a brief overview will be given to the high object oriented simulation API which
was written in the context of this thesis, with the python programming language.
The purpose of the library in general is to simplify the execution of simulations, but furthermore
includes the possibility to analyse the results and use visualization techniques to study the overall
behavior of the simulations during runtime.
At first some of the the different API classes and objects will be introduced,
followed by a small application in form of a grid convergence study.
REMARKS SRC ETC

\section{Parameter File}

The Parameter File contains every flag and parameter which is used during the runtime of the simulation.
During compile time all parameters are converted into a specific set of C-macros which are then compiled
into the cuda code. As a file format the open-standard format \textbf{Json} \footnote{Javascript Object Notation} is used.
An example of a Parameter file is shown in Listing \ref{lst_json}
The parameter file contains of two sections.

\begin{description}
\item[Conditions] Flags which are set to a constant value during runtime mostly zero or one. For example the boundary conditions and
                  interpolation methods. Not all flags have to be set for a simulation.
\item[Parameters] The parameter set defines all parameters which are used  for a simulation. In contrast to the conditions, each
                  parameter has to be defined in order to enable a proper execution.
\end{description}

\begin{minipage}{\linewidth}
\begin{lstlisting}[language=json, caption={'Example of a "parameter.json" file.}, label={lst_json} ]
{
    "conditions": {
        "all_periodic" :1  //periodioc boundaries in all directions
    },
    "parameters": {
        "BLOCKSIZE": 8,                //Defaultl GPU blocksize
        "STEPMAX": 200000,             //Number of timesteps
        "IC_NAME": "\"c2_1000\"",      // Initial condition files (not in use anymore)
        "RAYLEIGH":0,                  // Rayleigh number, not used in this thesis
        "DELTA_T": 0.0001,             // Time step
        "SOUND_SPEED_SQUARED": 400,    // Speed of sound squared
        "PRANDTL": 0.01,               // Prandtl number, not used in this thesis
        "GPU_ID": 2,                   // GPU id, there are up to 8 gpus in one system
        "EKMAN": 0.0001,               // Ekman number
        "NX": 64,                      // Resolution in x-direction
        "NY": 64,                      // Resolution in y-direction
        "NZ": 64,                      // Resolution in z-direction
        "LX": 1.0,                     // Length in x-direction
        "LX": 1.0,                     // Length in y-direction
        "LZ": 1.0,                     // Length in z-direction
        "LZ": 1.0,
        "RUN_NAME": "\"c2_1000\"",     //Name of .ekin output files
        "NUM_GPU": 1,                  //Number of GPUs used for the computation
        "SAMPLING_RATE": 5000,         //Number of samplings for data analysis
        "NX_D": "NX/NUM_GPU",          //NX for one GPU with threadings
        "NU": 0.0001,                  //Damping force
        "PM": 1,                       //Magnetic Prandtl number (not used)
        "KAPPA": 0.0001                //Thermal diffusivity (not used)
    }
}
\end{lstlisting}
\end{minipage}

\section{Generator Class}

The generator class is responsible for the generation of all initial data.
This means the initial conditions for all variables i.e. velocity and temperature
and the computation of interpolation and domain masks which are necessary for the different
immersed boundary methods.\\
During the execution of a simulation all precomputed arrays are stored within a
HDF5-File format, which is optimized for the storage and structuring of large amounts of data.
Furthermore the format simplifies the data exchange between the python API and the CUDA programm.\\
For the generation of data a generator object has to be initialized with a generator function,
there are currently implemented two different types of these functions.

\begin{description}
\item[Initial  Conditions] These functions simply generate the initial conditions for a certain flow problem, for example the Taylor-Couette
                            flow or a simple cylindric domain. The definition of the functions can be found in \textbf{pycurb.ic}
\item[Testcase] The test case functions extend the initial conditions by having the possibility to add certain forcing parameters into the time step
                for example a pressure gradient is necessary for the Poiseuille flow test case described in Sec. \ref{vali:sec_lpflow_setup}
                All defined functions can be found in pycurb.testcase.
\end{description}

Additionally it is possible to create a generator which takes the data from an old simulation. In listing \ref{gen_class_usage} the creation of a generator object is
exemplary shown. In a first step the generator is created by using a generator function. In the next step it is possible to define certain attributes
for example the radius of a cylindric fluid domain.\\

\begin{python} [caption='Generator class usage', label={gen_class_usage}]
import pycurb as pc #import the simulation API
import pycurb.testcase as tf #import testcases
import pycurb.ic as as ic    #import initial conditions

#create a generator object for hagen poiseuille flow in z direction
generator = pc.Generator.from_testcase(tf.cylinder_flow_z)
#set the default velocity profile, pressure gradient and radius
generator.add_option('SETV', True)
generator.add_option('PMAX', pmax)
generator.add_option('r', 1.)

#create a generator object for taylor couette flow
generator = pc.Generator.from_ic(ic.taylor_couette)
#set velocity, inner and outer radius
generator.add_option('SETV', True)
generator.add_option('ri', ri)
generator.add_option('ro', ro)
\end{python}

\section{Simulation Class}

An intance of the simulation class is the main object of the API and
necessary to execute a simulation. For the creation of a simulation object
the initial arguments are the filepath where all simulation data will be stored
and the json-file for the parameter setting.
Following the creation of the simulation object it is possible to alter all parameters and conditions
which were previously stored in the json-file. This gives the possibilty to create
simulations with different parameter settings on the fly, i.e. change the resolution of the numerical grid
to perform a grid convergence study.
Before the execution of the cuda code, the simulation object has to be bind to a generator object
with the \textbf{generate\_files} class function, in return the generator object will begin
with the data creation. The last step is execution of the cuda code which can be done with the \textbf{start\_simulation}
class function.  A minimal use case of the complete procedure is shown in listing \ref{gen_usage_nextone}.

\clearpage

\begin{python}[caption='Simulation class usage', label={gen_usage_nextone}]
import pycurb as pc
import pycurb.ic as ic

#create a generator
generator = pc.Generator.from_ic(ic.taylor_couette)
#create a simulation object
sim = pc.Simulation('data', 'parameter.json')
sim.parameter.set('NX', 128)
sim.parameter.set('NY', 128)
sim.parameter.set_condition('o2', order)
sim.generate_files(generator)
#start the simulation
sim.start_simulation()
\end{python}

\section{Usage Example}

Finally we want to give a usage example for a grid convergence study using
the Simulation API. The source code is shown in listing \ref{gcstudy_papi}.
In the example we simulate a Hagen-Poiseuille flow, for a more detailed discussion of this test case  we refer to section \ref{vali:section_hpflow_start}.
We start by importing the relevant modules from the API and defining all relevant constants like the
Reynolds number, $Re=100$ in this case.
Finally we create and loop over an  array which contains all resolutions on which we want to test
our fluid problem.\\
Inside the for loop we begin by defining a unique simulation path for each resolution.
The we create the generator with the pipe-flow settings from the \textbf{pycurb.testcase} module and
the options for the pressure gradient, the initial velocity field and the pipe radius.\\
Finally the simulation object is created, here we set all parameters like resolution and domain size,
the order of the finite difference scheme and the direct forcing method which will be introduced in section ().
The last step is to execute a single simulation.\\
%The given example is build to be executed on a single machine in order to further parallize the computations on
%more than one gpu an additional python script \textbf{worker.py} was written which is not yet included in the simulation API (see Appendix A(XX)).
%To parallelize the example some small adaptions has to be made. For each resolution the simulation path has to be stored into a textfile, which
%is than opened by the python script, furthermore the execution of the simulation has to be removed from the for-loop.
%Finally we start by generating all simulation data by executing the modified version of our script. Afterwards we start
%the parallelization script for which we have to specify the file including the simulation path and the number of gpus and cuda machines we want to use.

\clearpage

\begin{python}[caption='Grid Convergence Study Example', label={gcstudy_papi}]
import pycurb as pc
import pycurb.testcase as tf
import numpy as np
import os

def main():
    #define simulation parameters
    re = 100.
    pmax, pr = 4./re,  1./re
    lx, ly, order = 2.5, 2.5, 1

    #vary N from 16 to 256 with dN=16
    res = np.linspace(16, 256., 256./16)

    for rs in res:#iterate through resolution array
        #create filepath for each simulation
        var_path = os.path.join(method,'res_%i' % rs)
        sim_path = os.path.join(os.path.dirname(__file__), "data", var_path)

        #create generator with simulation settings
        generator = pc.Generator.from_testcase(tf.cylinder_flow_z)
        generator.add_option('SETV', True)
        generator.add_option('PMAX', pmax)
        generator.add_option('r', 1.)

        #create simulation object and set parameters
        sim = pc.Simulation(sim_path, "parameter.json")
        sim.parameter.set("PRANDTL", pr)
        lz = dx*(lx/rs)
        sim.parameter.set("NX", rs)
        sim.parameter.set("NY", rs)
        sim.parameter.set("LX", lx)
        sim.parameter.set("LY", ly)
        sim.parameter.set("NZ", nz)
        sim.parameter.set("LZ", lz)
        sim.parameter.set_condition('o2', order)
        sim.parameter.set_condition('SET_ZERO', 1)
        sim.generate_files(generator)

        #execute simulation
        sim.start_simulation()

if __name__=='__main__':
    main()
\end{python}

%The script creates a thread-safe queue and a number of threads, one for each gpu per machine.
%Each thread automatically fetches a job from the queue, than logs into the assigned cuda machine
%and executes the job on the given gpu.
%For time consuming simulations also a modificated version of the script was written where one thread
%can use more than one gpu, see Appendix A(XX).
