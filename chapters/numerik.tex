\chapter{Numerische Methoden}

\section{Einleitung}

Mit den in Kapitel () eingeführten Bewegungsgleichungen fehlt uns nur noch die Numerik um die zeitliche Entwicklung eines Models aus
gegebene Anfangs- und Randbedingungen zu berechnen.
Für die numerische Lösung von partiellen Differentialgleichungen eignen sich eine Vielzahl an Methoden, für die Berechnung auf
der GPU-Architektur erweist sich aber insbesondere die Verwendung von Finite Differenzen  auf kartesischen Gittern
als sehr effektiv (siehe Kapitel ()).
Während die Finite Differenzen für die räumliche Diskretisierung verwendet werden, wird für die zeitliche Diskretisierung
das Runge-Kutta Verfahren 3. Ordnung benutzt.


\section{Finite Differenzen}
-beispiel erste abl dann zweit
-stabilität konstistenz - konvergenz etc
-verfahren höherer ordnung
-peclet zahl
-upwind schema

\section{Runge Kutta}
-runge kutta richardson etc


\Section{Artificial Kompressibility
- bewegungsgleichungen
- druckterm diskussion
    -exkurs laplace gleichung



-beispiel rayleigh benard diskretisierung
