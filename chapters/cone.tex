\chapter{Inertial Waves in a Rotating Cone}

\section{Introduction}

Following the introduction and validation of the immersed boundary methods for no-slip conditions,
we now want to exemplarly investigate a fluid system using these methods
and see if we can reproduce some of the expected physical properties.\\
In relation to the research focus of the geophysical fluid mechanics research group there are a variety
topics of interest.
One area of research lies in the exploration of dynamo effects in geological and stellar system.
In particular this means the generation of magnetic fields by electrically conducting fluids on large scales.
In this thesis we will not consider MHD-equations.
However in general it is considered that the helicity of a fluid, given by
\begin{align}
    v curl v
\end{align}
is directly linked to dynamo action.
Therefore, beside inertial wave propagation we want to observe a possible large helicity.
The fluid experiment we want to investigate is the excitation of inertial waves inside a cone.
Due to the asymmetrie in z-direction and the possibility of inertial wave extinction
this system could be a candidate for a geodynamo. Furthermore it could be interesting to
study inertial wave excitaton by turbulence in this kind of setup.

\section{Theoretical description}



- Cone inertial waves desription
- introduction geodynamo
-magnetic field
- turbulence
- reference paper


\section{Numerical Viscosity}
