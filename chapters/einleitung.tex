\chapter{Einleitung}
\section{halleluja}
I was often using any of the available “lorem ipsum” generators on the web while testing different things in Latex until I discovered that the Latex distribution provides packages generating blind text, which is definitely more convenient. With just a few lines of code, these packages will generate paragraphes, even whole documents with sections, paragraphs of text, lists, etc.

The first package that I will introduce is the “blindtext” package. First the language option as well as the package have to be loaded. Make sure

\subsection{jawohl}
I was often using any of the available “lorem ipsum” generators on the web while testing different things in Latex until I discovered that the Latex distribution provides packages generating blind text, which is definitely more convenient. With just a few lines of code, these packages will generate paragraphes, even whole documents with sections, paragraphs of text, lists, etc.
I was often using any of the available “lorem ipsum” generators on the web while testing different things in Latex until I discovered that the Latex distribution provides packages generating blind text, which is definitely more convenient. With just a few lines of code, these packages will generate paragraphes, even whole documents with sections, paragraphs of text, lists, etc.
I was often using any of the available “lorem ipsum” generators on the web while testing different things in Latex until I discovered that the Latex distribution provides packages generating blind text, which is definitely more convenient. With just a few lines of code, these packages will generate paragraphes, even whole documents with sections, paragraphs of text, lists, etc.

The first package that I will introduce is the “blindtext” package. First the language option as well as the package have to be loaded. Make sure

The first package that I will introduce is the “blindtext” package. First the language option as well as the package have to be loaded. Make sure

The first package that I will introduce is the “blindtext” package. First the language option as well as the package have to be loaded. Make sure
u
