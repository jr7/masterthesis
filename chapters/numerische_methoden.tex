\chapter{Numerische Methoden}
\section{Einleitung}
\section{GPGPU Computing mit CUDA}
\section{Finite Differenzen Verfahren}
\section{Validierung}
\section{Immersed Boundary Methoden}

\newpage

\subsection{Einleitung}
Die bisher eingeführten Methoden eignen sich um auf einfachen Geometrien Simulationen durchzuführen.
Wenn die Ränder des Simulations-Gebietes nicht mehr mit dem numerischen Gitter übereinstimmen lassen sich die in Abschnitt () eingeführten Randbedingungen nicht mehr verwenden.
Das Problem lässt sich durch die Einführung eines an den Rand angepassten Gitter umgehen (s.Abb.), führt allerdings zu einem deutlich höheren Aufwand in den numerischen Berechnungen.
Zudem ist nicht eindeutig geklärt  unter welchen weiteren Performance-Einbussen sich das neue Gitter in der GPGPU-Implementierung verwenden lässt, da Bedingungen wie
z.B. Coalesced Readings  (s.Abbschnitt x) durch die unstrukturierten Daten deulich erschwert wäre.
Eine Alternative die in dieser Arbeit verwendet wird sind  sog. Immersed Boundary Methoden.
- Beschreibung allgemein verschiedene immersed boundary methoden direkt /exp/ imp etc
- hier einfache methoden wegen gpu



\subsection{No-Slip-Boundaries}
- zunächst gehen wir nur auf den fall einer der geschwindigkeiten ein bllabla
Folgende Verfahren formel mit kraftterm
weiterer abschnitt temperatur / no-flux

\subsubsection{Volume Penalization}
-allgemeinsch beschreibung paper zitat, zitat dipl. lüllff.
-idee dämpungsterm in der dgl
-problem instabilität

\subsubsection{Direct Forcing}
-Paper quote
-vgl volume penalization warum keine instablilität
-formeln
-implementierung

\subsubsection{Direct Forcing mit Volume Fraction}
-paper quote formeln
-implementierung beispiel

\subsubsection{Direct Forcing mit Interpolation}
-paper quote formeln
-implementierung beispiel

\subsection{Methoden-Vergleich und numerische Validierung}
Blabla
\subsubsection{Validierung mit MASA}
-validierung mit masa für alle verfahren oben.. cube /evtl zylinder?
-vegl. und argumentation ränder ehh auf null.

\subsubsection{Plane Poiseuille Flow}
\subsubsection{Poiseuille Flow im Zylinder}

\subsubsection{Zusammenfassung}


\subsection{No-Flux-Boundaries}

\subsubsection{'Variable Konduktivität'}










